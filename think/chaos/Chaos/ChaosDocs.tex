%! LaTeX
% Revisions:
% 970302: Page layout for various devices... --lct

\documentstyle[12pt]{article}

\voffset= -0.5in %\voffset = -4cm
\hoffset= -0.5in %\hoffset = -2cm
\textwidth = 6.5in  %\textwidth 160mm
\textheight = 9.5in %\textheight 10in
% LaTeX files have a top margin, headheight, and headsep from the 0,0 point.
\headsep = 0pt
%\headheight = 0pt
\topmargin = 0pt

\parindent = 0mm                                % Disable paragraph indenting
\parskip = 10pt plus2pt minus.5pt               % paragraph break

\def\undertext#1{$\underline{\smash{\hbox{#1}}}$}
\def\spell#1#2{\undertext{\bf #1} ({\sl #2\/})\ }


\begin{document}

\title{Chaos User Manual\\V2.3} \author{Sean A. Irvine\and James Conwell}

\maketitle

This is a mythical strategy game for up to eight human and/or computer
players. It is a game of justice and vengeance, of good versus evil,
of life versus death, of light versus shadow, as a group of wizards battle
ferociously to exterminate their opponents and achieve world
domination.

If you dare to compete, you will be faced with beasts of all
descriptions, from everyday animals like horses and lions; to monsters
of hideous proportions, from fire-breathing multi-headed pyrohydras to
extremely dangerous dragons that sail through the air hunting their
favorite food -- you. Creatures from the spirit world are also in
abundance; you may even be confronted with the devil himself. Say your
prayers if you do.

You will experience the most cataclysmic natural disasters ever to
occur in the course of human history. You may be drowned in
swift-moving floodwaters, burned beyond recognition in raging
uncontrolled hellfires, or be swallowed up by the very Earth itself in
horrific earthquakes.

With your own eyes you will witness corpses rising from the dead to
rejoin the battle. You will be struck with terror as your army
succumbs to massive piles of green slime and purple fungi. Even the
humble tree is not always what it seems.

Your army is not restricted to man alone. Your forces will be filled
out with orcs, elves, ogres, and a host of other races. You will
doubtless send many of these creatures to a premature death as they
blindly follow your every command. Only those wizards who use the
strategy of a Perfect General, possess the bloodlust of a Warlord,
make effective use of their Hired Guns and have the zeal to build an
Empire will triumph in the end.

You will utilize and be subjected to some of the most powerful magic,
sorcery, and necromancy known to the human mind. You may be cursed,
drained, and infected with disease. You may disappear into the
ethereal plane or vanish into a timewarp. You will twist time, garble
gravity, and rip holes in space as you criminally violate all the
laws of physics with the unbridled arrogance of the dictatorial
tyrant you really are.

You will cause untold destruction of a once pristine environment as
you destroy entire forests on a whim, saturate the air with
foul-smelling toxic fumes and unleash the apocalyptic fury of nuclear
devastation on all who oppose you.
 
The grapes of wrath never tasted so good.

To survive, you must strike a fine balance between defending your
wizard and attacking the other wizards. Only one wizard can rule the
world, but it can be a long time in coming, with up to eight wizards
chasing the same goal. Furthermore, the wizards may lose control of
their powerful magic and suffer the dreaded fate of mutual
annihilation; forever cleansing the world of disruptive magic!

\section {System Requirements}

Versions 2.3d and below may be played on any Amiga with 1 megabyte of ram
and OS 1.2 or higher.  Yes, even if you have an old 1985 A1000 with Kickstart
1.2 you can play this great game so long as you have 1 meg or more of ram.

The authors went to great lengths and much trouble to ensure compatiblity
with all the old Amigas.  We hope you enjoy the fruits of our labors.

Version 2.3d is the LAST version that will work on any Amiga with 1 meg of
ram.  ALL versions above 2.3d require at least 2 megs of ram, a hard drive,
at least OS 1.2 or higher and a Motorola 68000 CPU or higher.

2.3d = demo version.  2.3f = full version which only registered users can
obtain.  The demo version cannot load games, and stops at turn 18.


\section {Installation}

Just toss it on your hard drive somewhere.

This game requires that req.library be in your LIBS:
Its probably already in there, but if it isn't then we included it for you
in the libs directory of this archive.  Just copy it into LIBS: and you
are all set.

There are no other files that you have to mess with.


\section {The Real Stuff}

It should be unnecessary to read all the instructions to play the
game.  For those who are itching to start skip to Section~1.1 to see
how to load the game and then just follow your nose. For the rest of
you, the following describes how the game is played.

Each player (human or computer controlled) starts the game with a
wizard and a repertoire of spells. The idea is to be the last wizard
alive and to amass the greatest score. Once your wizard is killed you
are out of the game, so protecting your wizard is vital to your
survival. The options available at the start of the game allow for a
wide range of scenarios.

A game is organized as a series of rounds. Each round consists of
several stages, and each living wizard may partake in each stage of
each round.  The major stages are: spell selection, spell casting, and
movement.  Each stage is described in detail below. As well as these
main stages there are also a number of auxiliary stages performed
solely by the computer.

The game can be played almost entirely with the mouse and is icon
driven.  All the entities in the game have animated graphical
representations.  A text region at the bottom of the screen is used to
display other information.  Initially, you may be perplexed by the
wide range of spells available but after a few games you should become
comfortable.

The world once was comprised of a 7000 mile diameter sphere covered
with beautiful vegetation, crystal clear water and wonderful wildlife.
But that was before the 1000 year wizard war of Irvine vs.~Conwell;
the culmination of which resulted in the destruction of most of the
prime material plane of existence.\footnote {Human Beings commonly
  refer to ``The Prime Material Plane of Existence'' as ``The
  Universe''.  Other important planes of existence are the Astral and
  Ethereal planes of existence.} All that is left is a flat
rectangular board of square cells, measuring 17 across by 14 down.

The game is played on this rectangular board of square cells. You move
your pieces around the board using the mouse. The cost of a move
depends on the distance between the initial cell and the final cell.
It costs more to move diagonally than it does to go vertically or
horizontally.

It is important to understand there is a distinction between the
gaming environment supplied by the computer and the computer
controlled entities.  When a computer player moves it does not get
access to any information that you cannot get. In fact, the computer
bases it moves on less information than a human player.


\subsection {How to Win}
\begin{itemize}
\item Kill all enemy wizards.
\item Kill or block all independent generators.
\end{itemize|

A generator needs an empty space on which to generate creatures onto.
So if you can block all the spaces surrounding a generator, then it cannot
generate anything.


\subsection {Game Startup}

The game can be started from the Workbench or from the command line.
Some of the options can only be controlled from the command line. The
usual way of starting from the command line is to enter the command
{\tt chaos -t} when in the directory containing the chaos files. There
is an icon for starting from the Workbench. Installation is discussed
in Section~\ref{install}. You should ensure Chaos has a stack of at
least 50000 bytes. From the command line this can be done by typing
{\tt stack 50000} before starting chaos. Those who have received Chaos
on a self-booting disk need not worry about this matter.

These instructions are accurate for 2.3 of the game. However, Chaos
is a rapidly evolving program and new features are continually being
added and some instructions may become obsolete or inaccurate in the
future. Only registered users are guaranteed of being informed about
new releases.

After passing the introductory screens you will arrive at the first
setup screen. This screen allows you to select which wizards will be
playing, whether or not they will be computer controlled, the teams,
and their names. By default it starts with two human controlled
wizards on different teams. The `T' column is used for team selection.
The icons appearing in this column represent a team. By clicking on
them they move on to the next team in the sequence. If two wizards are
showing the same icon this means they will start on the same team.
There also exist spells which can alter the teams during the course of
a game.

Once you are satisfied with the setup click on `Start' which will take
you to the second initialization screen. On this screen you can
select the number of generators (default 4, maximum 99), number of
spells (default 20, maximum 99), and level of computer wizards
(maximum 9).

The wizard level influences only the allocation of spells to computer
players and does not influence the way the computer moves or casts
spells. Any wizard can improve their level rating by casting Level
spells during the game. Doing so increases the chances of getting
useful, rare and powerful bonus spells.

You should decide how many turns the game should last for. The default
action is forever which means the game continues until all human
players have been killed or one human player remains and all computer
players are dead. Alternatively, any number up to 999 can be
specified.  The number of turns remaining is displayed at the top
right of the screen as a series of tally marks. If these are shown in
yellow then you are into the last few turns.

You can also turn ``Texas Trash'em'' on or off.  Texas Trash'em allows you
to optionally trash (discard) one of your spells in addition to casting a
spell each turn.  This allows you to discard spells you don't like or
don't want.  In version 1.8 and below, computer players do not discard
spells.  In version 1.9 and up you MUST discard a spell each turn but
casting a spell is optional (unless there are 5 spells or less left in your
spell list in which case you don't have to discard if you don't want to.)

By default the game will operate in ``Classic'' mode. In this mode
only a limited number of spells are available. This mode is
particularly suited to neophytes. If you switch off this mode then the
full repertoire of spells is made available.

You may also select the number of artifacts that will be randomly placed
on the board. Allowable settings are 0 to 99.

Artifacts are like a regular creature object except it belongs to
NOBODY until someone moves adjacent to it.

An Artifact's owner is determined at the beginning of each movement phase.
It is determined according to who has the most creatures adjacent to it.
In the event that 2 or more players have an equal number of creatures
adjacent to it then it remains NOBODY's.

Inanimate objects and other artifacts do {\em NOT} assist in controlling
artifacts.

Currently there are only 4 artifacts:
Bolter Wall Artifacts
Dalek Artifacts
Robot Artifacts
Stone Golem Artifacts

Artifacts have 0 movement points.

Most magic spells can be cast on Artifacts so you can heal an Artifact
or increase its capabilities, etc.

If you play Chaos against human opponents who hide in a corner, cowering in
fear, then try playing the game with some artifacts.  The sight of a
nice, big, powerful artifact just sitting there waiting for someone to
claim it can be a strong incentive for them to charge out after it and
has been known to spark intense power struggles.

Hidden Artifacts:
If this is turned on then artifacts will be buried underneath
walls, trees, rocks and other inanimate objects.  These objects will have
to be destoryed in order to take posession of the artifact underneath.


You can also select the number of scrolls that will appear on the board.
Allowable settings are 0 to 99. A scroll is a spell lying on the board
which is free for the taking. Isn't that cool? Scrolls are magically
inscribed on extremely delicate, dry, brittle, paper. They are
therefore excessively fragile and even the tiniest amount of damage will
destroy them. The scroll rules should be set to one of the following:

\begin{enumerate}
\item Scrolls can be picked up by anybody.
\item Scrolls can be picked up by wizards only.
\item Scrolls can be picked up by anybody except flying creatures.
\end{enumerate}

Inanimate objects can never pick up scrolls.

There are normally only 3 ways to affect scrolls:
\begin{itemize}
\item By entering a cell that contains a scroll and surviving there until
   the end of the complete round. The complete round means until
   everyone has finished their turn for this round, including all
   wizards and independents.  If you survive in a cell containing a
   scroll until the end of the round then the scroll is magically
   added to your commanding Wizard's spell list.
\item Destroy them by shooting them.
\item Destroy them by Burying them with the Bury spell.
\end{itemize}

 
NOTE: Shooting the scrolls of Undead Elementals may produce Ungood side
effects. }8>
 
If you play Chaos against human players who just sit in a corner and
avoid conflicts with each other then try playing the game with some
scrolls laying around.  This can draw them out of their corner and
start up some fierce battles!


Scroll Delay:
Allowable settings are 0 to 6.

This controls how many consecutive turns you must control the cell before
you are awarded the scroll.  0 means you will acquire it at the end of
{\em this} round.  1 means you will acquire it at the end of the
{\em next} round and so forth.

If at the end of a round, you no longer control the cell then you must
start over again so if Scroll Delay is 6 and you lose control of the cell
after only 5 turns then you must regain control for 6 consecutive turns to
obtain the scroll.

You are permitted to move creatures into and out of the cell containing the
scroll without any penalty so long as you control the cell with a
scroll-capable creature at the end of the round.


Hidden Scrolls:
If this is turned on then scrolls will be buried underneath
walls, trees, rocks and other inanimate objects.  These objects will have
to be destoryed in order to take posession of the scroll underneath.


At this time you should not change the board size.

When you have made any desired changes, click on `Start' to begin the
game.

\subsection {Spell Selection}

For each human wizard in turn, the right of the screen is filled with
the currently available spells for that wizard. Information can be
obtained about the spells by reading about the spells in this manual
or by selecting the `?' gadget and then clicking on the spell you wish
to know about.  Or simply by pointing the mouse at the spell in question
and then pressing the `q' or `?' key on the keyboard.

A wizard can decide not to choose a spell by clicking
the return gadget, whence control will pass to the next wizard.

Assuming you wish to pick a spell, simply click on the desired spell.
During the casting stage you will attempt to use your selected spell.
Care should be taken in spell selection. Consider such things as the
range of the spell, the situation of the wizard, your position on the
board, and whether line of sight is required for the spell.  Just press `c'
(for Casting possibilities) while the mouse is over a spell and all the
possible places that you could cast that spell will be highlighted on the
board.

Once selected, a spell must be attempted.  If the spell is aborted, it will
still be considered used and will not appear on your spell list in the next
round.

If ``Texas trash'em'' was selected then you will also be required to
select a spell to be discarded (unless you have less than six spells
remaining). Such a selection must be made even if you choose not to
select a spell for casting.

\subsection {Spell Casting}

During spell casting, each player who selected a spell during the
spell selection stage will in turn get a chance to actuate the spell.
Computer controlled wizards will attempt to cast a spell at their
position in the sequence. In addition, each generator on the screen
might produce a creature. Certain other creatures also occasionally
cast spells.

For each player in turn the spell selected is shown on the bottom line
of the screen, together with the number of attempts and the range for
the spell. For example,

\centerline {\tt Gray Elf x1  Range=2}

means 1 Gray Elf is being cast, and the elf can be cast either
immediately adjacent to the wizard or one cell removed from the
wizard (just press {\tt c} to see all possible legal cells).
A range of zero means the spell can be cast simply by pressing
the left mouse button and there is no need to select a cell. Such
spells are often special powers for the wizard. The computer
automatically applies the spell to the correct wizard or creature.
Certain spells have a range of 15, these spells can be cast on any
appropriate cell on the screen and do not require line of sight.

Most spells have a fixed range, measured in a straight line from the
casting wizard. For instance, nearly all creatures have a range of one
and so can only be cast immediately adjacent to the wizard. All these
spells require line of sight. The line of sight is measured from the
centre of the wizard's cell to the centre of the destination cell.
You can see over blank space, dead creatures, pits, pools, and fires.
Sometimes line of sight can be obtained by waiting until an
appropriate time in the animation sequence.\footnote{To experienced
  players this is known as {\bf forcing.}} By pressing {\tt c} you can
ask the computer to show you all legal target cells.

Spells can be aborted by pressing the right mouse button. Aborted
spells still get used up, so try to avoid having to abort spells. If
you find yourself aborting spells often, you are probably not choosing
spells appropriate to your situation.

Most spells can be cast only on certain types of cell (the program
will notify you of any errors you make in selecting the cell). If you
make an error you are free to choose a new cell. For instance,
creatures can be cast on empty cells or cells containing dead
creatures, but cannot be cast on other creatures or inanimate objects.

After each wizard has cast, the generators may produce creatures on
any cell adjacent to the generator which is empty, a growth, or
contains a dead creature. Adjacent cells occupied by creatures or
inanimate objects will never be generated upon.

\subsection {Board Update}

The board update is handled entirely by the computer. The update
carries out activities which are not directly under any player's
control. During the update any or all of the following things might
happen:
\begin{itemize}
\item The growths (earthquakes, fire, flood, fungi, orange jelly,
  gooey blob, and green ooze) are permitted to grow. Each cell
  containing a growth may or may not grow. If a cell is going to grow
  it will do so in only one direction in any particular turn. The
  direction of growth is chosen at random. Earthquakes cannot `grow'
  diagonally. Some of the growths grow faster than others, orange
  jelly is particularly voracious while floods tend to be sluggish. If
  a cell doesn't grow it might dissipate instead. You may have some of
  your creatures buried, killed, or revealed in this growth process.
  Should your wizard be buried or killed you are considered dead and
  your game ends (an exception occurs in the case of a wizard having
  previously cast Lich Lord).
  
\item Any tempests on the board may move. A creature in the path of
  the tempest will be thrown to a new random location. On very rare
  occasions a creature will be killed as a result of the throw.
  
\item Any vortices may move. A creature in its path will disappear
  into the future.
  
\item Ropers get tired of constantly sitting in the same place.
  Sometimes they decide to teleport to a new location.
  
\item There is a chance that each magic castle and dark citadel on the
  board might collapse. If a wizard is inside the castle or citadel
  when the collapse occurs the wizard will be granted a bonus spell.
  
\item Any magic wood occupied by a wizard might collapse, resulting in
  a new spell for the wizard.
  
\item Each sleeping entity is examined and has a chance of dying or
  collapsing. The weaker things are more likely to be destroyed than
  the stronger things. The destruction will be complete and no trace
  of the entity will remain.
  
\item Each Haunt has a small chance of changing sides.
  
\item Each Spriggan has a small chance of falling asleep.
  
\item (When impurities are turned on) With a small probability an
  object might decide to change sides. This may work to your advantage
  or disadvantage.  It is even possible that a generator will become
  your friend and decide to emit creatures for you.
\end{itemize}

\subsection {Movement}

The movement stage is the most complex stage, since normally there are
a large number of creatures to be moved and each creature has a number
of possible moves. Each player is given a chance to move in turn. The
strategies employed during the movement stage will have the greatest
effect on your survival in the game. Movement involves not only the
shifting of creature positions but also combat.

Dead creatures, sleeping creatures, and growths cannot be moved. No
inanimates can be moved, although shadow wood trees and bolters may be
used to attack enemy creatures\footnote{Unlike creatures, most trees
  cannot attack other inanimate objects, or growths} in adjacent
cells.  Special objects like ropers and wasp nests are controlled by
the computer.

A player is able to move each creation once in each turn. To end your
turn click on the return gadget and control will pass to the next
player.  Players having cast the Move It spell in the current round
will get a second chance to move after all other players have moved,
but before the independents move.

A creature is said to be {\bf engaged\/} if it is adjacent to an enemy
creature and is unable to retreat. Engaged creatures cannot be moved
but may still be able to fight. How likely a creature is to be engaged
is dependent on its manoeuvrability and the number of adjacent hostile
creatures.

To move a creature simply click on the creature using the left mouse
button. Normally, the distance the creature may be moved will be
displayed in the text bar. Depending on the creature, it will be
flying movement or walking movement. There is a special cursor when
flying movement is active. To see all legal moves press the {\tt m}
key.

\subsubsection {Flying Movement}

For flying creatures you can pick any cell within the given range
except cells occupied by your team's creations. If the cell is empty
or contains a dead creature then the creature will move to the
selected cell, otherwise an aerial attack will be made on the selected
cell.

\subsubsection {Aerial Attacks}

An aerial attack results when a flying creature is moved onto a cell
occupied by an opponent. The flier will attack the current occupant.
If unsuccessful the attacker will remain at its current location. If
successful the current occupant will die and the flying creature will
be moved into the cell.

\subsubsection {Walking Movement}

A ground based creature can only move one cell at a time and you must
show the computer exactly how to move multiple cells by clicking in
each cell in turn. Many walking creatures can only move one cell
anyway.  For those which can move more than one cell you should select
a sequence of adjacent cells---you can abort the move at any time by
pressing the right mouse button. During the course of a move a
creature may become engaged in which case it will move no further.

\subsubsection {Combat}

Engaged creatures can take part in combat. You cannot attack your
team's creatures (but see ranged combat). Living creatures cannot
(normally) attack undead creatures, otherwise anything goes. Once
engaged, an attack is induced by clicking on the cell containing the
creature you wish to attack.  If successful, the defendant will be
killed and the attacker will move onto the corpse. For a successful
kill of a strong opponent you may be awarded a bonus spell.

It is possible to attack inanimate objects and growths, although you
cannot get engaged to them. Be wary of killing growths with buried
creatures beneath, as the buried creature will be revealed and will
probably be quite hungry.

\subsubsection {Ranged Combat}

Some creatures such as elves, dragons, and wizards carrying bows, have
additional attacks. Such an attack is indicated by the {\bf sights
  pointer}. Line of sight is required for most of these weapons. It is
possible to shoot your own creations or creations of your team
members, so choose your targets carefully. Occasionally it might be
desirable to kill one your own creatures due to insanity or disease
(especially if you are playing for points).

Creatures which posess the {\bf Archery} skill (such as the Wood Elf and
Gray Elf) do not need line of sight to shoot a target.  Their shots simply
arch over any intervening obstacles. 

\subsubsection {Wizard Movement}

Under normal circumstances the wizard can be moved like any other
creature.  However, wizards are capable of special movements. Wizards
can climb magic trees, enter castles, and ride certain creatures. For
convenience these activities are collectively called {\bf mounting}.
You cannot mount if you are engaged, otherwise mounting is effected by
simply moving the wizard onto the cell occupied by the entity you wish
to mount.

Dismounting is somewhat more complicated as the computer needs to
distinguish which of the two entities is to be moved. For trees and
castles the computer will assume the wizard. For ridden creatures the
default is the creature. To dismount from a creature the following
sequence should occur:

\begin{itemize}
\item Select the creature the wizard is riding.
\item Click on the `D' gadget at the top right of the screen (which
  will now be unghosted).
\item Move the wizard to a cell adjacent to the creature (wizards with
  high movement allowance or wings still only move one cell when
  dismounting).
\end{itemize}

Mounting is generally a good idea, as it hides the wizard from view
and gives the wizard additional protection.

\subsubsection {Ending Your Turn}

Once you have moved all creatures that you wish to, select the end of
turn gadget. Computer controlled wizards will move their creations at
the appropriate point in the sequence. After all wizards have moved
the independent creatures will be moved by the computer according to
the same rules. Finally, any wasp nests, ropers, and fungi on the
screen will attack adjacent opponents.

The round is now effectively over and the game returns to the spell
selection stage for the next round.

\subsection {Scoring}

The computer keeps track of a score for each player and this is
indicated by the computer displaying a ranking of all wizards and the
independents at the end of the game. There is a score and a life
column. The life column is the sum of the life force of a wizard's
creations excepting walls, castles, and certain other inanimate
objects. The current scores can be examined during the spell selection
stage using the menus.

Points are awarded for killing enemy creatures and wizards. The
stronger the creatures you kill the more points you get. Bonus spells
are awarded for killing particularly strong creatures. Points are also
awarded for successfully casting some difficult or unusual spells.
Points are even awarded for destroying or burying corpses.

\subsection {Information}

The `?' gadget can be used to get information about objects on the
board (at any time) and about spells (during spell selection). For an
object on the screen, a window is displayed showing various
characteristics of the chosen object. The `?' gadget will not reveal
any useful information when applied to a cloaked creature, unless you
own it.

The information screen for a wizard or a cell containing a mounted
wizard will also display the various power-ups held by the wizard as
icons.

Most creature's Combat stats affect their opponent's Life stat but
Some creatures, like the Mind Flayer, attack Intelligence or some
other stat. Sometimes a creature's Combat or Ranged Combat or Special
Combat will affect more than one of your Enemy's stats at a time.

Just press the Left Mouse button on either Combat, Ranged Combat or
Special Combat and all the stats that it attacks will begin flashing.

\subsection {Keys}

By pressing the number keys {\tt 1}--{\tt 9} the creations of the
corresponding player can be highlighted. Key {\tt 9} highlights the
independent creatures.  Creatures who have not moved are highlighted
in a cycling color.  Creatures who have moved are highlighted in red.
Allies are highlighted in blue.

Pressing {\tt q} for Query is another way to bring up the
information screen for a creature.  Once brought up you can press {\tt q}
again to quit the info screen or press any other key to continue on to the
2nd info screen.

Pressing {\tt c} for Cast during the spellcasting phase will show all the
legal targets for the spell about to be cast.  It takes into account all
the rules for spell casting such as range, line-of-sight, wizard powerups
and the exact type of thing in the target cell (growth, creature,
inanimate, etc.)

Pressing {\tt m} for Move shows all the legal moves that the currently
highlighted creature can make or if no creature is currently
highlighted then it shows the legal moves for the creature under the
mouse pointer.  `Moving' includes `attacking' so often the cells of
enemy creatures will be highlighted (because it is legal for you to
attempt to move there) even though you perhaps don't have a snowball's
chance in hell of really killing the creature (e.g. orc vs. stone
giant).

Pressing {\tt r} for Range shows all the cells that are within range of
the currently highlighted creature's ranged weapon or if no creature is
currently highlighted then it shows all cells that are in range of the
creature under the mouse pointer.  {\em Note\/}: this shows all the
cells that are within range not all the cells that you can actually
hit.  Many times the cells shown will be unshootable due to
intervening objects blocking line-of-sight or the fact that undeads
cannot be shot by living creatures.

Pressing {\tt s} for Shoot shows all the cells that can actually be shot
by the currently highlighted creature's ranged weapon or if no creature is
currently highlihgted then it shows all the cells that can actually be
shot by the creature under the mouse pointer. {\em Note\/}: this takes
into account line-of-sight and living/undead restrictions.

Pressing {\tt l} for Line-of-Sight shows all the cells that can be seen by
the currently highlighted creature or if no creature is highlighted then it
shows all the cells that can be seen by the the cell under the mouse pointer.
This is particularly useful when you are thinking of casting a spell that
requires line-of-sight and you want to see if your wizard can actually see
the intended target.  It is also good for checking out where enemy wizards
can/cannot see to so you can hide your creatures behind walls and things to
protect them from subversion/betrayal/abduction/sleep/braindrain etc.

Pressing {\tt $\langle$shift$\rangle$} in combination with {\tt m},
{\tt r}, {\tt s} or {\tt l} will {\it always\/} show the relevent cells
for the creature under the mouse pointer, regardless of any currently
highlighted creature.

\subsection {Menus}

The menus can only be used during the spell selection stage. They are
primarily concerned with loading and saving games, but it is also
possible to turn the speech on and off. The sound option is used to
switch off the sound while the independents move. Turn it off if you
want the game to go faster but be warned it is hard to follow what is
happening.  There is also a menu option to view the current scores.
The ShowCast menu item is used to control the graphical effect used
during spellcasting.  Turin it off if you want the game to go faster
but be warned it is hard to follow what is happening.

\section {\label{install}Regulations and Installations}

 The source code, documentation, levels and executables
are copyright \copyright Sean A. Irvine, James Conwell 1994,
1995, 1996, 1997, 1998, 1999 and may not be modified or distributed
without the author's permission.

Chaos will work under at least Workbench 1.2, 1.3, 2.04, 3.0 and 3.1 and
should work on anything subsequent. It requires an Amiga supporting PAL mode.
All Amigas manufactured since 1989 support PAL mode.  Any Amiga
manufactured before 1989 supports PAL mode if it has been upgraded with
an enhanced Agnes chip. (Enhanced Denise not needed).

If you live in the USA or Canada then you'll need to boot your computer
into PAL mode.  The best way to do this is to use the program "degrader",
though in this case you aren't going to actually degrade anything.
With degrader select "50 Hz", "50 Hz System" and "Survive Resets" and
your Amiga will be rebooted in PAL mode and will think it is in a European
country operating in PAL mode.

Normally a North American Amiga operates in NTSC mode 640x400 resolution.
PAL mode produces a resolution of 640x512.

Chaos makes use of the {\tt narrator.device} and {\tt req.library}.
As of version 2.2 Chaos no longer makes any use of {\tt ReqTools.library).

If the {\tt-t} option is used the {\tt translator.library} is also used.
Other zrequired files are the executable {\tt chaos} and the data files
{\tt pool.ssc} and {\tt chaos.prb} all of which must reside in the current
directory when Chaos is started.  There are various other fonts, music and
SoundEffect files which reside in the directories of the same name within
the current directory.

The {\tt -t} option forces translation of player names for the speech
and requires the {\tt translator.library}. A few other options are
supported, to find out about these use {\tt chaos -h} and then press
$\langle\cal A$-M$\rangle$ to return to the CLI screen.

\section {Future Additions}

We can only work on Chaos in our spare time but are always pleased to
hear suggestions for improvements from people who play the game.

Here is a list of features that may be added in the future:

\begin{itemize}
\item More spells (send me your favourite ideas, and pictures of
  wanted creatures).
  
\item Improved sound and sampled speech.
  
\item More flexibility in game options, such as allowing players to
  buy spells, and so on.
  
\item Level editor.
  
\item Amiga guide documentation.
  
\item HTML documentation.
\end{itemize}

\section {Known Bugs}

Chaos has been extensively tested with {\sl Enforcer\/} and in our
experience does nothing nasty. However, crashes may happen, if they do
it is most helpful to me if you can note carefully what was happening
when the crash occurred. There are a few less serious bugs within the
game which may be fixed in the future.

\begin{itemize}
\item {\sl Bolter Bug\/}: When a computer controlled Bolter gains a
  movement rating (e.g. by the Speed spell) it is able to move. A
  human controlled Bolter does not have this ability. Since this is
  rather amusing when it happens, we have no intention of removing it.
\item {\sl Reinstatement Bug\/}: When Dark Power, Vengeance, Decree,
  or Justice succeeds in killing a creature upon a corpse, the
  creature is killed but the corpse is brought back to life. This ``bug''
  is deliberately incorporated in the program.  Advanced players will
  exploit this fact in the obvious way.
\item {\sl Cat Lord Bugs\/}: Theoretically there should only ever be
  one Cat Lord. However, it possible to create more than one by a
  variety of methods: (i) using Alter Reality, (ii) using mutate,
  (iii) by waking a sleeping one, (iv) by raising a dead one, (v)
  using Request. Multiple Cat Lords will not cause the game to crash
  but may result in cats changing sides several times, moving multiple
  times, and other undocumented effects. We leave it to you to explore
  the possibilities.
\item {\sl Classic Bugs\/}: The classic mode is not perfect and it is
  possible that spells not normally available will occur. Known cases
  are the faun casting sleep.  The thundermare and iridium powerups
  are still possible. If generators are used then the bipedal generator
  will still generate creatures not normally available in classic mode.
\end{itemize}

\section {Registration}

If you register this game then you will automatically be an Official
Playtester and your name will be placed in the credits.  This means you'll
be famous once the Playstation, N64, DreamCast, Mac and PC versions are
released.

If you would like to register this fine game and recieve the full version
with loading, saving and unlimited turns capability then please send $35.00
US to:

If you are in North America, South America, Europe or Scandinavia and are
sending US $ Cash (That's actual money, not a check or money order) then
send it to:

\begin{tabular}{l}
James Conwell\\
16806 Carrack Turn\\
Friendswood, TX 77546-2311\\
USA
\end{tabular}

{\tt JamesConwell@yahoo.com}

Due to the lameness of the US banking system, European checks and currency
are nearly worthless here so _please_ send actual US dollars.


If you are in New Zealand or Australia, or if you insist on sending foreign
currency then please send it to:

\begin{tabular}{l}
Sean A. Irvine\\
5 Westra View\\
Tawa\\
Wellington\\
New Zealand
\end{tabular}

{\tt sairvin@xtra.co.nz}

Apparently the NZ banking system isn't as arrogant as the US system so they
apparently don't have the same problems with foreign currency as we have in
the USA.


Be sure to tell us if you want your registered version sent to you via
Email or snail mail.

\section {Equipment Used}

The following hardware and software was used during the development of
Chaos:

\subsection {Hardware}
Amiga 3000 with 25Mhz 68030, 225 Megs HD, 14 Megs RAM and 1950
Multisync Monitor for coding.
Amiga 2000 with 9 megs ram for testing 68000 compatability.
Amiga 1200 with 2 megs ram and 120 meg HD for testing AGA compatability.
Hewlett-Packard Laserjet 4M LaserPrinter with 6 megs ram for printing out
long program listings and documentation.
Perfect Sound Digitizer, Radio Shack Mixer, Alesis 3630
Compressor/Limiter, Telspec Multiline Teleconferencing Telephone
System.

Beginning July 13, 1998 a 4.3 gig hard drive began to be used.

\subsection{Software}
SASC 6.5, Cygnus Ed Professional 3.0, Devpac Assembler 3.14,
A68K Assembler, Audition 4, Protracker, JR-Comm,
Octamed Sound Studio Professional, Flashfind, Add36k, True Basic 2.0,
YAM 1.3.5, Miami 3.0, Ibrowse 1.2, AmFTP 1.5x, AmIRC 1.6
and lots of other stuff.
 
\subsection{Miscellaneous}
James Conwell has had the following utilities installed on his Amigas for
years and has used them every day and they work great and he recommends
them to everyone: CPUblit 1.0, Arq 1.78 (Animated Requesters), FastExec
(Loads Exec.library to fastram instead of chipram; speeds up multitasking
and interrupts, etc.)

\section {Acknowledgments}

Many people have made useful and not so useful suggestions about
Chaos.  Whenever possible we try to incorporate the suggestions of
ardent fans.  While most of the correspondence we receive is strongly
supportive, we are open to criticism as well. Naturally, we give
considerably more weighting to the suggestions and requests of
registered users, but will gladly listen to the opinions of anyone.

The following people have helped in many ways and Chaos would
definitely not be the program it is without their support (our
apologies for any glaring omissions): Mike Baron, Julian Gollop, Ingo
Holewczuk, Callum Irvine, Glen Irvine, James Levell, Nicole More,
Richard Romanowski, Stephen Smith, Michael Van Elst, Accolyte, Mikael
Kalms, Kalsu, Bryan Ford, Matthew Smith, Jonathan Rudolph, Joseph Conwell,
Jerry Conwell.

Req.library by Colin Fox and Bruce Dawson.

\TeXnical Assistance by Lance Tagliapietra.

32-bit Multitasking Operating System by Carl Sassenrath.

Hardware Design by Jay Miner {\em et.  al.}

\section{Technical Features}
This game was coded in a mixture of JC + Assembly code.  We use JC for the
main game and super high speed optimized assembly code for certain
time-critical things like the music and the ray tracing and certain math
operations and certain graphics operations.

Version 2.3 contains 313 spells.

If you have an 020 or higher processor then
this game features 50 fps animation in the quadrascopes and
quadspectrograms.  If you only have a 7 Mhz 68000 then you will only get
about 25 fps.

The sprite that highlights the current creature only runs at 25 fps no
matter how fast your computer is.  We did this on purpose because when it
moves faster than 25 fps it becomes unpleasing to the eye.

If you have an 020 or higher processor then the color cycling on the QUIT
and NEW GAME requesters moves at 50 fps.  If you only have a 7 Mhz 68000
then it is only about 25 fps.

The vector graphic pentagrams on the ABOUT requester rotate at 50 fps
on all Amigas.

The intro music utilizes 4 simultaneous channels of audio, each one at up
to 28,000 samples per second.

The introductory credits are displayed in 640x512 resolution with 8 colors
and smooth full-screen scrolling.

This game utilizes ray-tracing.  When calculating line of sight to a target
a beam of light is traced from the source to the target.  If the beam gets
through then you have line of sight.  If the beam is blocked then you do not
have line of sight.

Ray Tracing is an inherently slow process so if you only have a 7 Mhz 68000
you will notice a delay of up to .5 seconds when you push the ``l'' key to
see all the cells on the board that you have line of sight to while the
computer ray-traces all the possiblities.  If this is too slow for you then
please upgrade to a 25 Mhz 030 and then you'll be happy 8)

This game was programmed to be 100% OS friendly and Multitasking friendly,
however it will not work on graphics cards because
1) Most graphics cards are lame and don't have sprites.
2) This game contains certain high speed routines that write directly to
   your display memory in order to go 10x faster than using slow OS calls.
   This is totally legal and system friendly; it just isn't compatible with
   add-on graphics cards, so if you have a graphics card you'll need to
   play the game on your regular built-in chipset instead of the graphics
   card.
3) Neither of the authors owns a graphics card.

This game can be played on Amigas which are equipped with graphics cards
however no benefit will be derived from the presence of a graphics card.
It didn't make sense for us to make graphics that go 10x slower for regular
Amiga owners just so that those with graphics cards could go a bit faster.
Neither of the programmers of this game own a graphics card.  If you want
us to make a new ``graphics-card-aware'' version of the game then you'll
have to buy us a graphics card to work with.

Version 2.3 utilizes a main gameboard resolution of 320x256 pixels with 32
colors.  The text area at the bottom of the main screen is displayed in
640x256 resolution with 8 colors.  The information panels are also
displayed with 640x256 resolution in 8 colors.

Bill Gates does not profit in any way from the sale or use of this game.

\section*{Appendix: Spell Descriptions}

To find out about spells when playing the game the `?' gadget can be
used. Not all spells are equally likely to occur and some can only be
obtained as bonus spells or by meditation.

Chaos~1.5 and up features a file called {\tt chaos.prb} which contains the
probabilities for the various spells. These can be changed using a
text editor. Please read carefully the instructions in the file if you
wish to do this, and remember to save a copy of the original file.

{\bf  Abath}  A  variation  on  the unicorn having a somewhat enlarged horn
which  it  can use to induce severe bruising in an opponent.  The abath can
also be ridden by the wizard.  For many years the existence of the creature
was  the  subject of much debate, and only recently have any instances come
to life.

{\bf  Abduction}  Like  subversion,  abduction is an attempt to convince an
enemy  creature  that  its interests may better lie with allegiance to your
wizard.   Intelligent  creatures  are  more likely to see through this ruse
than  stupid creatures.  However, as abduction applies to a random creature
you  cannot  know  beforehand  the intelligence of the recipient.  Further,
because  the targeting of this spell is so imprecise it is possible that it
will be applied to one of your own creations.  Therefore, the spell is most
beneficial  when  you  have relatively few creations.  Line of sight is not
required  and  your chosen target can be anywhere on the board.  This spell
will  not  attempt  to  target  Artifacts, Scrolls, Wizards or corpses.  If
successful,  the  target will immediately come under control of your wizard
and no longer respond to the commands of its previous lord.

{\bf Achiyalabopa} A large low-flying bird with poor eyesight.  The history
of  this  creature  is  unclear, but it has evolved feathers which resemble
knives.    It  attacks  simply  by  brushing  against  the  opponent.   The
knife-like  feathers  are sharp enough to make a large number of serrations
on the hides of the opponent.

{\bf  Acid  Rain}  Causes  a  global  rain  storm  with droplets of extreme
acidity.   The  rain  drops  are a mixture of water, sulphuric acid, formic
acid,  nitric acid, phosphoric acid, carbonic acid, and surprisingly citric
acid.   The  droplets  also  contain  hydrazine and various other flammable
compounds.  Because of these flammable compounds acid rain will not put out
fires.   Further,  acid  rain  does  not harm creatures in general although
during  the  rain  it may be painful to them.  However, trees are extremely
vulnerable  to  this  type of rain.  Even the hardy magic wood cannot stand
acid rain.

{\bf Aerial Servant} An aerial servant is a semi-intelligent air spirit and
can  only  be  dimly  seen.   They have no effect in combat, but their high
vitality  makes  them  an  ideal  blocking  creature.  They can move almost
anywhere in a single bound and have reasonable agility.

{\bf  Agathion}  A  powerful and helpful race which have descended from the
stars.   Although  appearing  in  humanoid  form, this is not their natural
form.   Their true form is unknown.  The agathion are not physically strong
but  can  cast  a  limited number of spells which they perform on their own
accord.   They  are  known  to  be  able  to cast bless, cure, haste, magic
shield, and vitality.

{\bf  Air Elemental} A powerful elemental best kept well clear of by mortal
creatures.  Upon its death the air elemental becomes a tempest.

{\bf  Alliance}  Force  another  wizard to enter into an alliance with you.
The  target  wizard  joins  your  team  and that wizard's creations will no
longer  engage  your  creatures.   You  can even force several wizards into
alliances  with  you by using multiple alliance spells.  When allied with a
computer  player,  the computer wizard will not cast harmful spells on your
wizard.   The independents are FIERCELY independent and don't want to be in
an  alliance  with  you.  (or anyone else for that matter.) So don't bother
trying it on them.

{\bf  Alter Reality} This spell can be cast on nearly any creature provided
it  is  not  dead.   The  spell  will  cause  the  creature  to  undergo  a
metamorphosis  to  a  new  creature  type.   This  spell can turn orcs into
dragons  and  vampires into ogres.  The result is random.  The spell can be
cast on creations of any wizard.

{\bf  Amphisbaena}  A  two headed snake about twice as powerful as a python
(which  makes it pretty weak actually).  Like the python, it kills its prey
by constriction.  Most creatures need not be fearful of this creature.

{\bf  Animate}  This  spell  causes  several different inanimate objects to
become  living  creatures.   The spell can be cast on any player, but it is
usually the best option to cast the spell on yourself.

{\bf  Apple  Wood}  This  spell allows a wizard to produce two apple trees.
While  useful  blocking  objects  the  trees  have  another use.  Any owned
creature adjacent to an apple tree will have its magical resistance boosted
by  one  unit  per  round.   The  trees cannot be used for meditation or in
combat.

{\bf Arborist} Special training for the wizard in the cultivation of trees.
A wizard having this training is able to cast trees next to each other.

{\bf  Archery} The target is given the archery powerup.  This spell is only
meaningful for objects having ranged combat.  It will have little effect on
other  creatures.   Further, there is little point in casting this spell on
creatures  (like elves) which already have this ability.  Ideal targets are
dragons,  bolters,  and wizards with bows.  After receiving this spell line
of sight will not be required for ranged combat.

{\bf Arctic Wolf} A wolf from the extreme cold regions.  A horrid carnivore
that  will  eat  whenever  possible.  Because they normally inhabit a harsh
environment they are very hardy.  They do not fear the undead.  They do not
fear  dragons  either.   They  attack  multiple  characteristics  and enjoy
ripping off the legs of their prey.

{\bf  Armour}  Armour can be used to boost the defences of any creature.  A
good  candidate  to  armour  is  your  wizard.   Armour adds a third of the
maximum  life  to  the  selected  creature.   You can apply armour to enemy
creatures  if  desired.   Line of sight is not required and the spell works
over the entire world.

{\bf  Aviary}  This is a like the general purpose generator but is somewhat
weaker  and  only  produces  a  limited  number of races.  It is capable of
generating  most of the bird species.  Casting such a generator will ensure
that  you  always  have  a  fresh  supply of flying creatures even if not a
strong army.

{\bf  Azer}  A  creature  which  in  appearence resembles an orc.  They are
masters  of  fire and are hot to the touch.  They show a lack of compassion
for  most  things,  and  although  they can control fires at will they very
rarely feel any motivation to do so.

{\bf  Baboon} The baboon is not particularly useful.  All its qualities are
somewhat less than those of the gorilla, except for its superior agility.

{\bf  Ball Lightning} An energized ball of lightning which severely damages
the  target's  combat ability.  Inflicts 6 points of damage to the target's
combat  ability  with  any  excess damage being transferred to the target's
life  force  at  a  2:1  ratio.  Thus using this spell on a creature with 1
combat  point  would  result  in its combat being reduced to 0 and its life
force being reduced by 10 points.

{\bf  Banderlog}  A  variant  of the baboon which is somewhat weaker than a
gorilla.   They can help keep the numbers up in a battle but are little use
as a single agent.

{\bf  Bandit}  A  human  mercenary  that  generally  behaves as a thief and
general  rogue.   Often  they  enjoy the taste of battle and will fight for
nothing.   They  quickly  become  bored  from  lack  of  action and are not
particularly  reliable or trustworthy.  This said, they enjoy mingling with
other  creatures  of  all types, forever on the lookout for quick financial
gain.  Maybe they are related to lawyers.

{\bf  Basalt  Golem} A form of golem made from basalt.  They are marginally
more  intelligent  than the usual stone golem.  One advantage is an ability
to  move considerably faster than other golems, since basalt holds together
better  when moving at speed.  However, the main advantage is an ability to
gaze at other creatures and almost paralyze them.

{\bf  Basilisk}  The  basilisk  is a reptilian monster whose slow metabolic
process  allows  it  only  slow  movement.   It  has  powerful  jaws  and a
penetrating  gaze  which  many foes find unsettling.  They are stubborn and
strong.

{\bf Bat} Only colour distinguishes the bat from the falcon or the vulture,
the bat having the least abilities of the three.  Although the bat is weak,
flight  allows  it to be a persistent menace to slower moving ground-ridden
species.

{\bf  Battle  Cry}  Causes  your  wizard  to  produce  a  sonic  shock wave
consisting  of  an  order  to battle hard.  The sound is guaranteed to stir
your  creatures  into  action.   All  your combatants will attack at double
strength.   The  voice  is  so strong that creatures will exceed the normal
maximum for combat if necessary.  The effect lasts only for the duration of
the  current  round.   The  spell  is most useful when you have many weaker
creatures and want one quick chance to wipe out many of the opposition.

{\bf  Betrayal}  Betrayal  is an attempt to convince an enemy creature that
its  interests  might  be  better served by changing its allegiance.  It is
essentially  a  weak  version  of subversion, as the convincing is only one
way.  That is, although the creature may be convinced to change sides there
is  little  chance  that  it  will decide upon your lord as its new leader.
However,  the spell is still useful because it works over a longer distance
than  subversion  and  no  wizard  likes  any adjacent creature to suddenly
become an enemy.

{\bf  Biohazard}  Releases  a deadly but selective toxin into the air which
will  cause  infections to run amuck in the wounds of the enemy.  All enemy
creatures will be slower at recovering from any wounds they receive.

{\bf Bipedal Generator} This is a like the general purpose generator but is
somewhat weaker and only produces a limited number of races.  It is capable
of  generating  most of the weaker bipedal races.  Casting such a generator
will  ensure that you always have a fresh supply of blocking creatures even
if not a strong army.

{\bf  Bird  Lord}  A  bird lord is a king of the avian species.  Often bird
lords  are mistaken as eagles.  Bird lords can heal any wounds they receive
at  a  phenomenal  rate.  They are amongst the fastest of all creatures and
are  well  learned.   Most magic will have little effect on them.  The bird
lord  possesses adequate combat skill and can easily flee from danger.  The
biggest  boon  of  the bird lord is its ability to fire an extremely acidic
salvo an immense distance.

{\bf  Bless}  The  bless  spell  is  used to improve the overall morale and
strength  of  a  wizard  and  creations.   Every  creation  of  the casting
wizard's, including inanimate objects and growths receives a small boost in
life.  Bless spells are best cast when you have many creations or when your
wizard  is  starting to founder.  You cannot bless the creations of another
player.

{\bf  Blue  Dragon}  The  blue  dragon  is  very  dangerous and can emit an
electrical  discharge  across  ten cells given line of sight.  Blue dragons
normally  have  little  to  do with other species and a summons can only be
obtained by meditation.  A blue dragon is stronger than a red dragon.

{\bf  Bodak}  A Bodak is a human that has become corrupted by venturing too
close  to  the  abyssal planes.  They appear in giant proportions and fight
bare  handed  even  though  most  of  them do carry a weapon.  Their combat
ability  is  similar  to  that  of  a  true giant, but they retain only the
stamina of a real human.

{\bf Boil} Causes a very specific cadence of microwaves to cover the world.
They  are  at  a  precise  value  which  causes  large  bodies  of water to
evaporate.  Any floods and pools will be destroyed.

{\bf  Bolter}  A  piece  of  wall augmented with a very high energy psionic
arrow  launcher.   The bolter can fire one arrow per turn with deadly force
over a medium distance.  Line of sight is required.

{\bf  Brain Boost} This spell can be cast on any creature and exponentially
increases the number of synaptic connections in the mental faculties of the
chosen  target.   The  spell  is  best used to protect a wizard against the
brain  drain  spell.   This  spell  increases  the target's intelligence to
maximum.

{\bf  Brain  Drain}  This spell will play on the mind of the chosen target.
The  spell  induces  torment  in  the  brain  of  the target causing severe
synaptic  damage.   For  weak  minded creatures the spell is usually fatal.
Damage  caused  by  a brain drain spell may be recoverable by a brain boost
spell.   This spell can be cast on enemy wizards.  This spell inflicts five
points  of  damage to the target's intelligence; if this results in a value
less than zero then the target dies of stupidity.

{\bf  Brown  Bear}  A  very  strong and skilled animal, the brown bear is a
tough  match for a lion.  Like the lion the brown bear is very ferocious in
attack.  The brown bear is not particularly agile and once engaged the bear
must  usually  fight to the death.  A brown bear can be killed by the rarer
but more powerful grizzly bear.

{\bf  Bury} Allows a wizard to bury up to eight corpses and/or scrolls from
anywhere  on the board.  Line of sight is not required.  Once buried a dead
creature  cannot be raised from the dead.  Therefore, by burying corpses of
dragons,  giants,  and  so  on,  you can prevent other wizards from raising
them.   The  spell  is also useful for clearing space for walls, trees, and
meditation  chambers  or  for  destroying  scrolls that you don't want your
enemies to grab.

{\bf  Camel}  The  camel  can be ridden by the wizard and is useful in this
respect.  As well as normal attack the camel can spit at nearby targets and
although  the spit is not potent it can sometimes be the difference between
life and death.

{\bf  Cat  Lord} The Cat Lord will assist a wizard when required but is not
particularly  suited  to  combat.   However, as master of all cats, the Cat
Lord  has  some  important  abilities.   When  first cast the Cat Lord will
immediately take all cats under his control.  He is also able to summon his
own lions, jaguars, and leopards.

{\bf  Centaur}  The centaur has the head and torso of a man and the abdomen
and  legs  of a horse.  A centaur may be ridden by the wizard.  The centaur
carries  a  bow which often comes in handy.  The centaur is superior to the
horse in most respects.

{\bf Chaos Lord} A wizard casting this spell will be completely enlightened
to   the  entire  universe  and  become  essentially  immortal.   A  wizard
possessing the level of potency induced by this spell will not need support
creatures,  mounts,  nor undeads.  Such a wizard may freely go hand to hand
with  a  demon and still expect to win easily.  All other players will need
to  forget  their  petty  squabbles if they are to defeat you.  If you have
this  spell  on your list it should be used immediately, no matter what the
circumstances.   This  spell  induces a state more powerful than the highly
sought  after  Touch of God spell.  Besides giving the wizard full strength
in all vital statistics the wizard will also receive the triple power up, a
bow,  wings,  fire  shield, flood shield, earthquake shield, reflector, pox
shield, and quickshot.

{\bf Charm} All creatures on the board will feel a tug towards your wizard.
Those  of  lesser  intelligence  may  be  so swayed by your words that they
change to your side.

{\bf  Clay  Golem}  As  the name suggests these unintelligent creations are
made  of  clay  and are therefore can take much punishment.  They are quite
susceptible to magic attack but are somewhat weaker than the stone golem.

{\bf  Cloak}  The  cloak  spell  can  be  used to place a shroud around any
creature  on the board including:  living, undead, flying, walking, ranged,
and  wizards.   A  cloak cannot be applied to inanimate objects or growths.
The  spell  can  be  cast  on  creatures  not owned by the caster including
sleeping  creatures.   The spell can be cast anywhere in the world and does
not  require  line  of  sight.   Once cloaked information about the cloaked
creature cannot readily be obtained.  Any cloaked creature becomes at least
twice  as  hard  to  hit because it is not easily seen by its opponents and
this  is  the  main  reason  for  using  the cloak.  The cloak spell can be
reversed by the reveal spell.

{\bf  Coercion}  A mental assault against all the other wizards.  The other
wizards  will  have  difficulty staving off the onslaught and they will not
have  time  to  select a spell in the subsequent round.  While most wizards
will  shake  of the effects of the spell within one round, some wizards may
suffer for several rounds.  This spell is useful in the early stages of the
game and gives you a chance to get ahead of the other wizards.

{\bf  Combat}  This  spell provides the recipient with advanced weapons and
tactical  skills.   The  result  is a very mean fighting machine capable of
killing  very  strong  opponents in a short time.  Even the humble halfling
becomes a deadly weapon upon receiving this special training.

{\bf  Command}  Command  increases  the  ability  of a wizard to direct its
creations.   In  particular,  creatures  will find it easier to escape from
conflicts if required to do so.

{\bf  Confidence}  Places  a blanket of calmness over your creations.  Your
living  creatures will no longer fear the undead and will be able to attack
the undead.  The effect lasts for three rounds.

{\bf  Consecrate} A spell similar to Dark Power but affecting a wider area.
Choose  the  centre  and  this spot and all cells immediately adjacent will
suffer  two  points  of  damage  to  their  magic  resistance.   Useful for
destroying  groups of weak undead creatures.  For some unknown reason, this
spell  must  be cast directly on a creature of some type; a corpse or empty
cell won't do.

{\bf  Convert}  The  target  creature will be converted from an undead to a
living  creature.   Does not affect any other stats.  Up to 2 creatures may
be converted.

{\bf  Crimson  Death}  A  pale  vapourous  creature named after its colour.
There  is  nothing  particularly  special  about the creature and it is yet
another creature from the spirit world.

{\bf Crocodile} Although the crocodile is stupid, any living creature which
approaches  one is a bigger fool.  The crocodile is a good match for a lion
or  a  bear, having a high endurance.  Usually the crocodile can sidle away
from danger if desired.

{\bf Crystal Ball} A highly desirable spell which removes the need for your
wizard  to  have  line of sight in order to cast a spell.  With the Crystal
Ball,  your wizard can "see" everywhere.  Range restrictions still apply as
usual.

{\bf  Cure}  All  creatures  have felt the ravages of the pox at some time.
This spell is useful in ending a major plague.  All creatures infected with
the pox will be cured of the disease.

{\bf  Curse}  The  curse is a highly desirable spell.  It is best cast when
the  world  is full of many enemy creations and your position is reasonably
secure.   Curse  will  suck  2  points  of  life  from all enemy creatures,
growths,  and even inanimate objects.  Sleeping creatures are also drained.
Any creature whose life is exhausted by the curse will be killed and points
credited to the casting wizard.

{\bf Cursed Sword} Works over the entire world and does not require line of
sight.   The  recipient  of  the  Cursed Sword will lose 5 points of combat
(will not go below 0) and will suddenly not be able to attack undeads.

{\bf  Dalek} This well known time-traveller can be a useful addition to any
campaign.   Their  sole purpose in life is the extermination of other forms
of  life and they are reasonably adept at carrying out this task.  However,
daleks are made of relatively cheap and weak metal.

{\bf  Dao} A genie of particularly evil alignment.  This genie will, on its
own  accord,  curse the enemy in your name.  Dao are also known to make use
of the drain, invert, pox, and slow spells.

{\bf  Dark  Citadel} The dark citadel provides a safe haven for the wizard,
and a place for meditation.  There is a chance of the citadel collapsing at
each  turn,  if  this  should happen while the wizard is inside a new spell
will  be  granted.  It is possible to make use of dark citadels conjured by
other  wizards  provided  they are not already occupied.  Only a wizard can
make  use  of  a  citadel.  A dark citadel can be destroyed by an extremely
determined attacker.

{\bf Dark Power} Dark power causes the target to be consumed by madness.  A
high  magical  resistance is required to escape the effect.  3 attempts are
granted,  each  of  which  inflicts  2  points  of damage to the creature's
magical  resistance.   Dark  power does not require line of sight and works
over the entire world.  Dark power cannot be applied to inanimate objects.

{\bf  Dark  Wood}  Dangerous  undead trees similar in ability to the shadow
wood  but more indiscriminate in what it attacks.  The trees are controlled
by  the  computer  and  will attack absolutely anything nearby except other
undeads.  Up to 6 trees may be cast.

{\bf  Dead  Revenge} This extremely rare spell allows a wizard to haunt the
opposition  even after death.  When a wizard protected with Dead Revenge is
killed  a  generator appears in the place where the wizard was killed.  The
generator will produce independent creatures.

{\bf  Death  Bringer}  A powerful spell which indiscriminately strikes down
one  creature  on the board.  Wizards are immune to the powerful magic used
in this spell, but creations of your own wizard are not.

{\bf  Demon}  The  demon  is  the  most  feared of all creatures.  They are
incredibly  strong,  exceptionally  intelligent,  and  designed for combat.
Anything less than a vampire best keep well clear of a demon.  Magic is the
best  approach  in  dispelling  a demon, but even this is a difficult task.
Thankfully, demons are very rare.

{\bf  Demonic  Touch}  Allows a wizard to call upon Satan to instill a long
term  fear  in a chosen target.  The recipient will no longer be able fight
effectively,  even  if  it  previously  received  advanced combat training.
Indeed, the target can only regain combat ability via subsequent retraining
in combat.

{\bf  Depth}  Allows all spells (except creatures) to be cast with infinite
range and without line of sight.  This is a very desirable spell because it
enables  a  wizard  to  cast  lightning  bolts  and  similar weapons at any
opponent.   Inanimate objects like trees, walls, growths and wasp nests can
also be cast in this manner.

{\bf  Derro} The derro are a degenerate race of metallic complexion.  Their
origin  is  uncertain.  Their stronghold is in subterranean relams and they
are  not  commonly  seen  above  ground.   Their  style  of fighting is not
particularly  suited  to open combat but they are persistent and not easily
toppled.

{\bf Destroy Wall} When cast this spell destroys every piece of wall in the
world  by  causing localized matter disturbances in areas containing walls.
This spell is only available through meditation.

{\bf Devastation} When you are tired of the current battles cast this spell
for  a  clean  start.   This  spell  essentially  wipes  the  slate  clean.
Everything  except  wizards  and  generators  is instantly destroyed.  Even
mounted  wizards  will be exposed.  However, this spell should be used with
some  caution.   It  is  best  to use this spell when you have several good
creatures  on your spell list with which you can repopulate the world after
the  devastation.   The  use of this spell is bound to cause outrage on the
part  of  other  wizards,  which  is  all  the  more reason to use it.  The
mechanism  of  this  spell  has  never been properly explained, but wizards
known  to  have cast it speak of a euphoric state of mind immediately after
the  devastation.   It  has  been conjectured that it involves some sort of
dimensional  shift.   This conjecture (which we note has no direct evidence
for it) is so vague that it is not really helpful.  Several wizards believe
that no adequate mechanism for this spell will ever be possible, and indeed
it may not be a computable quantity.

{\bf  Dire  Wolf} A huge variety of the normal wolf.  Single wolves are not
very  useful  but  if  collected  into  a  group  of  about  six or used in
conjunction with other species the dire wolf can be very useful.  They also
work well with goblins.

{\bf  Discard}  A  powerful  magical  spell which burrows into the minds of
wizards  and  tampers  with  their  memory.   Its usual effect is temporary
amnesia,   and   the   wizard  will  find  most  spells  have  suspiciously
disappeared.  These spells will eventually return, sometimes after a single
turn.  A single invocation will induce this effect in all other wizards.

{\bf  Disection}  Causes  a virtual knife to pass through the entire world.
All objects including wizards and generators will feel its presence.  It is
like half the life force of the world is drained away in an instant.  It is
known  to cause enfeeblement, nausea, and general light headedness.  At the
conclusion of the spell all objects (including those of the caster) will be
at  half strength.  Creatures with wound recovery will eventually heal, but
for  others the damage is irreversible.  The end result is that for several
turns afterwards kills will be a lot easier to make.

{\bf  Disrupt}  This  spell causes a momentary lapse of vision in all enemy
creatures.   It  can  prove  fatal  to  weak  minded  creatures  and causes
permanent  damage to the cerebral cortex of those that survive.  This spell
currently does not affect wizards, growths, inanimate objects or scrolls.

{\bf  Double}  This  is  a  sought  after  spell which gives the wizard the
ability  to  make  two  attempts  with each subsequent spell, for a limited
time.   It  does  not  work  with  a  few taxing spells like magic wood and
justice.  It is most valuable near the start of the game when building your
forces.

{\bf  Dragon  Nest}  This  is  a  like the general purpose generator but is
somewhat weaker and only produces a limited number of races.  It is capable
of  generating  most  of the dragon species.  Casting such a generator will
ensure that you always have a fresh supply of strong creatures.

{\bf  Drain}  A  powerful  spell  that wearies all the opposition.  All the
opposition  creatures  will  feel inexplicably tired and will not feel like
moving very far.

{\bf  Dread  Elf}  Not all elves are equal.  The dread elf is an elf who by
previous  feats has high status among all elves.  They benefit greatly from
this  revered  status and always carry the very best bows.  Their extensive
battle experience makes them much more hardy than the usual elves.

{\bf  Drelb}  The  drelb  is  very difficult to distinguish from a standard
ghost.  The touch of a drelb is chilling to mortal flesh, more so than that
of the ghost, but they are spiritually weaker than the ghost.

{\bf  Dual Earthbind} The target of this spell loses their wings and can no
longer  fly.   This  has  no  effect  on  movement  points.  2 attempts are
granted.

{\bf Eagle} The eagle can attack from great distances and move with extreme
agility.   Most  land-ridden  creatures  have  trouble  defending an aerial
attack  by  an  eagle.   Other birds such as the vulture and the falcon can
give the eagle a testing fight.

{\bf  Earth Elemental} A muddy concoction from the depths of the earth, the
earth  elemental  is  incredibly strong and is shunned by living creatures.
Its destruction normally results in a strong earthquake.

{\bf  Earthquake}  A  destructive  movement  of the ground initiated by the
wizard  or  resulting from the death of an earth elemental.  The earthquake
will  instantly  kill any creature in its path.  It will also crumble walls
and cause other damage.  There is no way to stop its damage.

{\bf   Earthquake  Shield}  A  special  shield  only  for  the  wizard,  to
permanently protect the wizard from being killed by an earthquake.

{\bf  Elephant}  The elephant is very strong and attacks with its trunk and
by  stomping.   Wizards  are not normally permitted to ride elephants.  The
strength  of  the  elephant makes it useful for attacking several opponents
simultaneously.

{\bf  Elf Boots} A pair of boots made by some very learned elves.  They are
a  perfect  fit  for  your wizard and will make your wizard one of the most
agile creatures on the board.

{\bf Elm} A rock and stock standard elm tree, great to sit under and listen
to the birds, just provides a little variety to the scenery.

{\bf  Emerald Dragon} The emerald dragon is a battle hardened green dragon.
They  have  exceptional  combat  ability.  Such a creature is normally only
available by having a green dragon successfully make seven kills.

{\bf  Exorcise}  Exorcism  is  used  to  remove an evil presence but is not
always  successful.  Line of sight is not required.  It can only be applied
to  undead  objects.   The chosen subject will suffer 6 points of damage to
its magical resistance.

{\bf  Eye for an Eye} Gives special coverage to all your creations.  Should
any  of  your creations be killed or otherwise destroyed in combat then the
killer  will  also  be  struck  dead.   The effect lasts for 2 kills and is
cumulative  with  multiple  Eye for an Eye spells.  Thus if you have triple
and cast this spell the effect will last for 6 turns.

{\bf  Falcon}  Falcons  are very swift and can cross the longer axis of the
visible  world  in  just three moves.  Because of this incredible range the
falcon  is  most  useful for mounting long range aerial assaults.  A falcon
can only be granted through meditation.

{\bf  Faun}  The  faun,  or  satyr as they are sometimes known, is a highly
intelligent  creature with a high magical resistance.  They are rather slow
and  not  particularly skilled in combat but have a fair chance of escaping
from  a  conflict if desired.  Unfortunately, once wounded they cannot heal
themselves.

{\bf  Fir} A plain and normal fir tree.  It has no special effects so don't
bother trying to find any.

{\bf  Fire} Fire is very dangerous once conjured.  A fire continues to grow
until  the conjurer is killed or until it is put out by other means.  Fires
can  burn  any  creature,  living  or  undead  or  dead,  or any wizard not
possessing  a  fire  shield.  Fires even burn the creatures of the conjurer
and can even burn the conjurer.  Fires can also burn down castles, citadels
and  even  walls,  although  this  usually  takes  considerable  time.  The
destructive  ability  of  a  fire  exceeds  that  of any creature including
dragons  and  vampires.  Fire can be put out by active effort of creatures,
each cell of fire has a energy comparable to that of a faun or troll.

{\bf  Fire  Bat}  A bat which has been influenced by the underworld.  Their
constitution is similar to that of the imp, but they are not able to attack
full fledged undeads.

{\bf  Fire  Demon}  The  fire  demon  is even more deadly than the standard
demon.  They are incredibly strong, exceptionally intelligent, and designed
for  combat.  In addition to the powers of the standard demon they can lash
out with their forked tongue.  Thankfully, they are exceptionally rare.

{\bf  Fire Elemental} A powerful creature possessing many of the properties
of  a  real  fire.   Living  creatures shun the elemental fire and will not
attack it, as it appears to them to be an undead.  Such elementals are best
disposed  using  magic.   However,  even  in  death the fire elemental is a
menace as it leaves behind a real fire.

{\bf Fire Shield} The fire shield can only be used by the wizard.  The fire
shield  gives  a  wizard immunity from fire, but not from dragon breath, or
heat orientated magical weapons like magic bolts.  Once cast it is unlikely
the fire shield will ever become damaged.

{\bf  Fireball}  Causes  the wizard to emit a ball of fire which will cause
magical  damage.   Anything can be shot provided it is in range and line of
sight is available.  Be careful not to shoot your own creations.

{\bf  Floating  Eye}  An unusual creature with the appearance of a floating
human  eye.   The  eye  induces a hypnotic trance in its enemies but is not
particularly strong.

{\bf  Flood} A flood behaves like a fire but tends to be short-lived.  That
is, a flood will often dissipate within two turns.  Any creature, living or
undead,  may  be  drowned  as  a  result of a flood.  Flood can also damage
creatures belonging to the casting wizard.

{\bf  Flood  Shield} A special shield, only available for the wizard, which
protects  the  wizard  from drowning in floods.  One a wizard gains a flood
shield it is unlikely to ever become damaged.

{\bf Fly} The target of this spell is given wings and can fly.  This has no
effect on movement points.  2 attempts are granted.

{\bf  Free}  Allows  the  wizard  to  liberate  one creature.  The selected
creature  immediately  becomes  an  independent.  If cast on an independent
creature, the creature will go into a trance.

{\bf Free All} Causes every creature on the board to become an independent.
This  is a useful spell if your opponents have guarded themselves well.  It
is guaranteed to get all the wizards running scared.

{\bf  Freeze}  A  powerful spell that will temporarily cause disorder among
all  creations  of the chosen target.  The enemy attacked will be unable to
move  or  partake in combat for the duration of the round, although you may
remain engaged if in combat.

{\bf  Generator}  The  generator  spell allows a wizard to cast a generator
identical  in  appearance  and  operation  to those used by the independent
creatures.   This  spell  is  only available by meditation.  It is unlikely
that  you  will  ever  get  one  of  these  spells.   Like  the independent
generator,  this generator will randomly produce creatures belonging to the
casting  wizard,  but unlike independent creatures, once created the wizard
takes  responsibility  for  their  movement.   Once  cast  the generator is
essentially eternal.

{\bf  Ghast}  Ghasts  are  akin to ghouls in both appearance and abilities.
They are most useful in blocking the onslaught from living creatures.

{\bf  Ghost}  The ghost is the spirit of a deceased evil human.  Ghosts can
fly  but  do  not  move particularly quickly.  Any ghost is best avoided by
living  creatures  but  ghosts  have  a  fair  chance of being destroyed if
attacked by another undead creature.

{\bf  Ghoul}  Once  a  human, these undeads now feed on others.  Ghouls are
terribly  cunning,  despite their low intelligence, and can cause paralysis
in lesser creatures.

{\bf  Giant  Beetle}  The giant beetle is a pesky creature that does a good
job  of  hindering  an  enemy  onslaught  until more powerful defenders can
arrive.   A giant beetle has an excellent chance of escaping conflict as it
is  very  agile.   Giant  beetles  are easily killed by even the weakest of
creatures.

{\bf  Giant  Rat}  The  giant  rat,  or  super  rat, is a quick and evasive
creature  with  minimal intelligence and magical resistance.  The giant rat
is  a  danger to orcs, horses, elves, halflings, beetles, goblins and ogres
but  is  only a minor hinderance to other creatures.  The bite of the giant
rat is poisonous.

{\bf  Giant  Spider}  A  menacing  hinderance  to the opposition.  Although
unlikely  to  be  fatal  to  any opposition the spider takes some effort to
kill.   This  spider  will  also  lay  web  as  it moves.  This spider eats
scrolls.

{\bf Goblin} No campaign is complete without a goblin.  The goblin is among
the  better  fighters of the common bipedal living species.  The goblin can
fight as well as the ogre but is slightly weaker in its defence.

{\bf  Goblin Bomb} No campaign is complete without a goblin.  The goblin is
among the better fighters of the common bipedal living species.  The goblin
can  fight as well as the ogre but is slightly weaker in its defence.  This
goblin carries a grenade and upon its death the goblin releases the grenade
which subsequently explodes causing damage to all adjacent cells.

{\bf Golden Dragon} A very intelligent creature that rules amongst the good
living  species  with  the exception of the immensely rare platinum dragon.
The  golden  dragon  can fight in a similar vain as the stone giant but has
the  gift  of  fire breathing and flight which makes it extremely deadly to
any  creature or wizard that attracts its attention.  The golden dragon can
kill some of the lesser dragons.

{\bf Gooey Blob} Gooey blob is particularly useful and can be conjured some
distance  from  the wizard provided there is line of sight.  The gooey blob
then  grows much like a fire does.  No part of the gooey blob can be moved.
Although  the  gooey blob has no intelligence it will never harm its master
or any of its maters creatures.  However, the gooey blob quickly grows over
enemy  creatures preventing them from moving but not actually killing them.
Buried  creatures  can  be  recovered  by  killing  the gooey blob on them.
Rescued  creatures  will  have  half  their  initial  life  remaining  when
recovered  no matter how much they had when they were buried, so their life
on recovery can be more or less than what it was when they were buried.  If
an enemy wizard succumbs to the blob that wizard dies.

{\bf  Gorilla}  Gorillas  are  a  good  middle of the road animal and are a
worthy  adversary  for most bipedal living creatures such as trolls, ogres,
and  goblins.   The  more violent animals such as the lion and grizzly bear
can  kill a gorilla.  The gorilla has only mediocre intelligence but fights
with vigour.

{\bf  Gravity  Sphere}  Causes  the  wizard  to  create  a  dense sphere of
neutrons,  much  like a miniature neutron star, which causes the gravity to
increase  exponentially.   Indeed,  gravity  becomes  so strong that flying
movement  becomes impossible.  All creatures which can normally fly will be
forced  to  move like any other creature.  The sphere will eventually decay
and  normal  flight  can then resume.  A typical instance will last for six
turns,  however, if several wizards cast this spell in succession the delay
could be much longer.

{\bf Gray Elf} The gray elf is very similar to the more common wood elf but
posesses  accelerated  healing  ability.   They carry the best bows and are
able to make accurate shots over a greater distance than the Wood Elf.

{\bf  Green  Dragon}  The  green  dragon  is the weakest species of dragon,
having  a  strength  similar  to that of a crocodile.  Of course, the green
dragon  is  far  more useful than the crocodile, as it can move further and
more  importantly  can  use a breath weapon.  The breath weapon consists of
chlorine  gas  and  is quite deadly.  The dragon will recover from injuries
but not as quickly as red and golden dragons do.

{\bf  Green Ooze} Green ooze is very similar to the gooey blob and it grows
in  the same manner.  The green ooze cannot be moved in the same way as you
move  your other creatures.  Instead it grows on its own accord.  Creatures
which  succumb  to  the  ooze  can  be recovered by killing the ooze on the
creature concerned.

{\bf Grizzly Bear} The grizzly bear is an extremely powerful animal capable
of  killing  a  lion  or  a  brown  bear.  Like most aggressive animals the
grizzly  bear  has  a  low magical resistance.  If reincarnated the grizzly
bear becomes an ogre mage.

{\bf  Gryphon}  A  large flying beast which can be ridden by the wizard who
conjured  it.  The gryphon has a good stamina and magical resistance but is
not  particularly agile.  The gryphon is not a particularly skilled fighter
but  can  move a considerable distance enabling a mounted wizard to quickly
escape from difficult situations.

{\bf  Halfling}  These  little  guys  are  extremely  magic resistant.  The
halfling is typically a hard-working, orderly, and peaceful citizen similar
to a human, although much smaller.  They usually fight with swords although
some  have  been  known  to  carry  bows.   While they fight well, and with
determination,  they  have  little  stamina and quickly become exhausted in
battle.

{\bf  Harpy}  The  harpy  is  a cross between a vulture and a female human.
Harpies  are  fast  and  can  mount  aerial attacks thereby avoiding direct
engagement.   Using  vulture  claws the harpy can rip into the hide of most
beasts and can outfly would be attackers.

{\bf  Haste}  A  spell used to urge your creations on to greater speed.  It
boosts  the  speed of all your creations by one unit.  This is particularly
useful after another player has cast Drain.

{\bf Haunt} A variant of the spectre.  Haunts are very difficult to control
and  will  occassionally  switch  sides for no apparent reason.  It follows
that haunts should be used with extreme caution.

{\bf  Hidden Horror} For all intents and purposes this is a halfling with a
singular  important difference.  Upon its death the hidden horror becomes a
red  dragon  (unless  it  is set to reincarnate, in which case it becomes a
golden  dragon).   The  halfling  is typically a hard-working, orderly, and
peaceful  citizen  similar to a human, although much smaller.  They usually
fight  with swords although some have been known to carry bows.  While they
fight  well,  and  with determination, they have little stamina and quickly
become exhausted in battle.

{\bf  Hide}  Causes  your  wizard  to disappear from the game for some time
(typically  about  five rounds).  During this time you will be unable to do
anything,  you  will  not  be able to cast creatures.  However, you will be
able  to  move any creatures you have left on the board.  The spell is most
useful  when  your  wizard  is  in  a tight spot.  You reappear at a random
position.

{\bf Higher Devil} A higher devil is summoned from from the pits of hell by
the  wizard.   They  somewhat  resemble  demons  in  their  appearance  and
abilities  and  are generally shunned by other creatures including undeads.
Once  cast  the  devil  is  very faithful to the wizard and will obediently
carry out any tasks set with no complaint.

{\bf  Horror}  A  powerful  form  of  sorcery  which  enables a creature to
replicate itself upon death by combat.

{\bf  Horse}  The  horse  may be ridden by the wizard and is useful in this
regard  for  carrying  the wizard out of a tight spot.  The horse, however,
best avoid conflicts and magical attack as they are not particularly suited
to combat.

{\bf  Hybsil}  An  antelope centaur appearing in similar form to a centaur.
They  are weaker than a true centaur but fire poisonned arrows.  The hybsil
can also be ridden by the wizard.

{\bf  Hydra} These slow moving multiheaded beasts are very hard to kill and
are  extremely apt at fighting.  The are not particularly stupid either but
have  great  trouble escaping from a conflict once initiated.  The hydra is
strong enough to hold most undeads at bay for several turns.

{\bf  Hyperclone}  An  extremely  potent  spell  which will replicate every
single creature controlled by your wizard.  Each clone will be functionally
equivalent  to the original.  Clearly this spell is most effective when you
already have a large number of creatures.

{\bf  Ice  Breath} Causes a wizard to breath out an icy beam.  The icy beam
can  only  be  resisted  by  magic.   A  creature with insufficient magical
resistance will die of hypothermia.

{\bf  Imp}  They imp is a very low form of devil and is not greatly feared.
Although the imp is not undead, it can attack undeads with ease.

{\bf  Invert}  The invert spell causes strange fluctuations in the strength
of creatures.  Normally strong creatures will suddenly become very weak and
conversely  weak  creatures will become strong.  The spell affects only the
creatures of the selected target.

{\bf  Iridium  Horse}  A extraordinarily rare super enhanced form of horse.
This  spell is even rarer than the powerful thundermare.  As a general rule
this spell does not appear on spell lists (you are about as likely to get a
generator  on your spell list).  Under normal circumstances an irdium horse
only  occurs when a thundermare kills a red, green, blue, or golden dragon.
The  iridium  horse  is  the  ultimate in mounts.  It is strong, swift, and
deadly  in  combat.   Not  only  can  it  attack  in  similar  vein  to the
thundermare, but it also has a powerful iridium stare which affects several
statistics  of  the  target  simultaneously.   The irdium horse also freely
attacks  undeads  even though it is not undead itself.  The nightmare pales
in  comparison  to this awesome beast.  Even demons are known to be wary of
this  animal.   The  iridium  horse  represents  the  most  powerful  mount
available and a wizard fortunate enough to have an Iridium Horse commits an
extreme faux pas if he does not ride this beast.

{\bf  Irvine's  Invulnerability}  Created  by  the ancient wizard Irvine to
stave  off  the  deadly attacks of his arch-rival King Conwell who seems to
have  had  a  fondness  for catapulting cows at him when he wasn't looking.
This  spell uses extremely powerful but rather unstable magic to render the
caster  invulnerable  to any and all physical attacks.  After several turns
the   instability   becomes   too  great  and  the  spell  evaporates  into
nothingness.

{\bf  Jaguar} This jungle predator is very dangerous despite being slightly
weaker  than the lion.  The jaguar can outrun a lion and can even match the
speed of some flying creatures.

{\bf  Jann}  The weakest of the genies.  This genie is able to cause random
occurrences  of  orange jelly, gooey blob, and green ooze.  A jann can also
boil water and quench fires.

{\bf  Joker}  A  powerful  spell  that  causes your entire spell list to be
randomly permuted.  When the joker is used, the chances of getting rare and
otherwise impossible to obtain spells improves.

{\bf  Juju  Zombie}  An  extra  powerful  zombie  that attacks a creature's
magical resistance rather than their life force.

{\bf Justice} A spell which attempts to make a creature see the evil of its
ways.   The  feedback generated is often enough to kill the target.  Can be
applied  to wizards, does not require line of sight and has infinite range.
2  attempts  are  granted, each of which inflicts 3 points of damage to the
creature's magical resistance.

{\bf  Kill}  An  extremely  potent spell only available by meditation.  The
chosen  target,  living  or  undead,  will  instantly drop dead.  The spell
simply  switches  off  the  creatures  brain.   Wizards  are  immune to its
effects.  Kills are best used on strong undeads such as demons.

{\bf  King Cobra} The king cobra has very little to recommend for it, aside
from the fact that its poisonous bite causes permanent damage to any victim
stupid  enough  to get close to it.  Every creature is capable of killing a
king  cobra,  even  creatures as low as bats, dire wolves, vultures, fauns,
elves,  and pegasii can kill a king cobra in a single blow.  The king cobra
is  also stupid, has minimal magical resistance, and is sluggish.  Assuming
the  cobra  survives  an  attack,  it  has  a  good  chance of escaping the
conflict.

{\bf  Leopard}  This predator is very ferocious although not as strong as a
lion.   They  can  outrun  nearly  any land-ridden creature and some flying
species.  The leopard offers considerable combat assistance to any party.

{\bf Lesser Devil} A lesser devil is summoned from from the pits of hell by
the  wizard.   They  somewhat  resemble  demons  in  their  appearance  and
abilities  and  are generally shunned by other creatures including undeads.
Once  cast,  the  devil  is very faithful to the wizard and will obediently
carry out any task with no complaint.

{\bf  Level}  Improves  your  wizard's  chances  of  getting  unlikely  and
difficult  spells  as  bonus  spells.   The effect is cumulative.  The more
level spells cast the better the chances of getting good spells.

{\bf  Lich}  This spell can only be applied to creatures on your team.  The
creature  concerned instantly becomes an undead but retains all its current
abilities.

{\bf  Lich Lord} This spell is for the wizard.  If your wizard is killed in
combat  the  wizard  is  reinstated in the game as an undead.  The powerful
magic  used in this spell usually produces a horrific, mutated copy of your
wizard as a side effect.

{\bf  Lightning}  An easy spell which causes the wizard to emit a lightning
bolt.   The  bolt  can  travel  up to four squares and has both magical and
physical components.  Line of sight is required.

{\bf  Lion}  The  lion  is  a  ferocious  beast  once provoked and can take
considerable  punishment from a foe.  The lion is amongst the faster of the
land-ridden  creatures  and has above-average intelligence.  The lion comes
into  its  own  in  combat  as it is a very proficient attacker with a good
ability to escape conflicts when things are not going its way.

{\bf  Magic  Attack}  This spell can be cast on an enemy creature or wizard
and  will  lower  the life force of the target, in extreme cases the target
may be killed.

{\bf Magic Bolt} A plain magical attack which works against any object.  If
used on a dead creature the corpse will be destroyed.  Will attack both the
magical  resistance and the life force of the chosen target.  Requires line
of sight.  Best used against creatures with low magical resistance.

{\bf  Magic Bow} A magic bow which can only be given to the casting wizard.
The  bow  will allow the wizard to shoot an arrow six squares each time the
wizard  moves.   In  addition  the  wizard will be able to attack and shoot
undeads.  The bow can be used in conjunction with other magical weapons.

{\bf  Magic Castle} The magic castle is functionally equivalent to the dark
citadel.   It  provides  a  safe  haven  for  the  wizard,  and a place for
meditation.   There  is  a chance of the castle collapsing at each turn, if
this should happen while the wizard is inside, a new spell will be granted.

{\bf  Magic Glass} Allows the wizard to produce four pieces of magic glass.
To the opposition the glass has the strength and consistency of a wall, but
to your own creatures the glass will allow line of sight and ranged combat.
It  is  the  ideal  substance  behind which to locate creatures with ranged
combat  or  even  to  hide your wizard.  One nice thing about this spell is
that you don't need Line of Sight to cast it.

{\bf  Magic  Knife}  A  knife  for  the  use of the wizard that has magical
properties  rendering  it  effective  against  undeads.   May  be  used  in
conjunction with a magic sword or bow.

{\bf  Magic  Shield}  A  shield  which can be used to increase a creature's
magical  resistance.   This  quite frequent spell should be applied to your
wizard  and  your  strong  creatures.   The  shield  remains in force until
damaged  by  magical  attack.  Applying a shield to a creature with maximum
magical resistance will give enhanced protection.

{\bf  Magic  Sword}  A  sword for the use of the wizard which increases the
wizards  combat  ability.   The  sword  is  magical and enables a wizard to
attack  undeads.   The  sword can be used in conjunction with other magical
weapons.

{\bf  Magic Wand} Wow!  What a great spell!  Each Magic Wand increases your
casting  range  by  1  square  cumulatively.   Any wizard who could somehow
obtain  2  or  3  of  these  would  be  awesomely powerful.  The Magic Wand
increases  the  casting  range of not only magical spells, but of creatures
too!

{\bf Magic Wood} Magic wood is very useful even though it cannot fight like
a shadow wood.  The biggest boon of the magic wood is that a wizard may use
such  a  tree for meditation.  When the tree collapses the wizard occupying
the tree at the time is granted a new spell.  Unlike castles the trees will
not  collapse  unless  occupied.   The  tree  also  provides  a wizard with
shielding  from  the watchful eyes of the independents.  Magic trees can be
destroyed  by  combat and some magical weapons.  Any wizard may make use of
any  magic tree provided it is not already occupied.  Each magic wood spell
can produce up to eight trees.

{\bf  Mana Battery} An energetic source of magical and psionic energy.  Any
wizard  adjacent  to  such  a  battery will regularly gain new spells.  The
battery is long life and none are known to have ceased operation by natural
causes.   Naturally,  independent creatures tend to loathe these batteries,
as they invariably result in a worse life for the independents.

{\bf  Manticore}  The manticore is one of the most powerful creatures which
may  be  ridden by a wizard.  The manticore has the advantage of being able
to  fire  spikes  from its tail as well as take part in direct combat.  The
manticore  also  has  a  high magical resistance.  The only drawback to the
manticore is its small brain.

{\bf  Marid}  The  most powerful genie.  A marid is able to perform quite a
number  of spells which are usually advantageous to the controlling wizard.
If granted the marid should probably be keep in a secure position.

{\bf  Mass  Morph}  A  powerful  spell  best  cast  on  an  enemy with many
creations.  In extreme circumstances, you might cast this spell on your own
wizard.   Mass  morph  changes  all of a wizards creatures into magic trees
which can then be used for meditation by any player.  No wizard is affected
by the spell.

{\bf  Mass Resurrect} A spell which brings all visible corpses back to life
as standard living beetles.  While beetles are not particularly useful this
spell does get rid of all the corpses which are visible.

{\bf  Materialize}  Are you sick of battling with creatures from the astral
planes?   Are you tired of your living creatures being slaughtered wantonly
by the undead?  Then this is the spell for you.  All undeads will instantly
become living creatures attackable by all creatures.

{\bf  Meddle}  A  wave  of  death  passes  over  than  land,  followed by a
rejuvination.   Some have likened it to a turning of the world.  The result
is  that  all creatures pass to their reincarnation.  The usual result is a
slight  weakening  of the world, since most creature reincarnate downwards.
An additional side effect is that all creatures lose any current ability to
reincarnate.   Things which cannot be reincarnated (like inanimate objects)
are unaffected by this spell.

{\bf Meteor Storm} Causes rocks to fall from the sky to random locations on
the  board.   Any  creature  hit by a rock is likely to be severely damaged
often resulting in death.

{\bf  Mind  Flayer} The tentacle arms of this happy looking killer suck the
intellect  of  nearby  opponents.   The mind flayer itself is very weak and
will probably only make a single attack before dying itself.

{\bf  Miner  Willy}  A  memorial  to the greatest miner of all time.  Miner
Willy  of  Jet  Set  Willy and Manic Miner fame.  This brilliant edifice to
Willy  gives  extra kudos to your team since it shows you feel so confident
in  your position as to have time to cast relatively weak spells.  However,
statue  Willy  is  not  entirely useless and he makes a useful obstruction.
This spell is also a good choice for discarding.

{\bf Mount} Works over the entire world and does not require line of sight.
The  recipient creature immediately becomes mountable and may be mounted by
the wizard.

{\bf Move It} An order to your creations to make an extra move.  This spell
is  useful  when  you  want  to  quickly assert your dominance on the game.
After  each  wizard has moved you will be given a second chance to move all
your  creatures.   The extra move will occur in the same round as the spell
is cast.

{\bf Mud Man} A piece of animated mud, taking a vaguely humanoid form which
bears  a  resemblance  to  the earth elemental.  They attack by hurling mud
balls at their opponents and are not easily damaged by swords or arrows.

{\bf  Mutate}  An  extremely  potent  spell,  similar  in function to Alter
Reality,  but affecting every creature of a given player.  The spell can be
used  on strong opposition to make it weak, or on yourself if you have many
weak creatures.

{\bf  Necropotence}  A magical spell that effectively blocks psionic forces
for  an indeterminate amount of time.  Other wizards will be unable to cast
magic  spells  for  a  limited  amount  of time.  It will not prevent enemy
wizards from casting creatures.  Your wizard, however, will be able to cast
spells  of  both  types.  Some wizards may regain the ability to cast magic
spells after a single turn, for other wizards this may take longer.

{\bf  Neo-Otyugh}  A  weird  tripodal  scavenger.  The neo-otyugh has three
tentacle arms one of which can lash out a considerable distance.  They have
a tough hide and can survive persistent attack.

{\bf  Nightmare}  Also known as a demon horse or hell horse, nightmares are
creatures  from  the  lower  planes.   They  can, however, be ridden by the
wizard.  They attack with their hooves which are burning hot.

{\bf  No  Grow}  Works  over  the entire world and does not require line of
sight.   Up  to  10 pieces of growth can be altered in such a way that they
will stop growing.

{\bf  No  Mount}  Works  over the entire world and does not require line of
sight.   The target creatures immediately become unmountable and may not be
mounted a wizard.  Up to 3 creatures may be made unmountable.

{\bf  Nonarchery}  Works over the entire world and does not require line of
sight.   2  attempts  are  granted and each will remove the archery ability
from the chosen target.

{\bf  Nuke} Nuke allows a wizard to invoke a local nuclear disturbance in a
chosen cell.  Anything in the cell will be instantly destroyed.  The chosen
cell will remain radioactive for the duration of the game.  Anything trying
to  enter or shoot into a radioactive cell also dies instantly.  Due to the
magical protective aura surrounding a wizard, Nuke will not succeed against
a  wizard  or anything that a wizard is riding on or meditating in.  Thus a
Nuke  will  succeed against a generator but will fail against a horse being
ridden by a wizard.  Nukes can only be obtained by meditation.

{\bf  Ogre}  The  ogre is similar in ability to the gorilla, the troll, and
the  goblin.   Like  these  other  creatures  the  ogre is very useful in a
supporting role in a major offensive but best avoid lone conflicts.

{\bf Ogre Assassin} The Ogre Assassin is a battle hardened ogre.  They have
been specially trained in a full gamut of lethal combat techniques and form
a  dependable  attack creature where there is a need to get in quick and do
damage.

{\bf  Ogre  Mage}  Some ogres receive special training in the use of magic.
These  ogres  enjoy  casting  fireballs at opponents over a great distance.
They  are  also  trained  in  the  medical arts.  They are identical to the
standard ogre in appearance.

{\bf  Ogre Warrior} The warrior ogre is a battle hardened ogre.  They fight
with  considerably  more  vigour  than  the standard orge and can take more
punishment  before  they will succumb to exhaustion.  They have little time
for magic.

{\bf Orange Jelly} Yet another growth.  Orange jelly grows very quickly but
is  correspondingly weak and even an orc has some chance of killing a piece
of  jelly.  Use an orange jelly when you want to quickly clog up an area of
the board.

{\bf  Orc}  The orc is typical of many exceedingly common species in having
very  limited abilities.  Free orcs band together in tribes but those under
a  wizard  have  deserted their tribes and pledged loyalty to their wizard.
Orcs are easily slain and are only of minimal use.  Their greatest asset is
their  frequency.  They hate elves and will kill them whenever possible but
would rather avoid the stronger ogre.

{\bf  Orcs}  Sometimes  when  an orc is cast he will con a few friends into
joining  your side.  Up to eight orcs in total may be obtained.  Although a
single  orc  offers  little,  a  group of eight is very useful for blocking
generators or in the early stages of the game.

{\bf  Paradigm Shift} Wizards generally maintain their forces of undeads by
continual  mental  domination  of the objects concerned.  This spell tweaks
the  minds  of  all  wizards so that all the undeads once again become only
potentials.  This means all undeads currently on the board will be removed.
However,  it  cannot  cleanse  them  completely  and  all such undeads will
reappear on their controlling wizard's spell list.  Indepedent undeads will
be  utterly  destroyed.   Normally  living creatures which have been raised
will  also  suffer the same fate.  The undeads of the caster are not immune
because  of the wide area pulse this spell creates.  The spell is best used
when  you  have few or no undeads of your own and is particularly effective
when there are many independent undeads.

{\bf  Passage}  A special ability for the wizard, which enables a wizard to
pass  through  any  distance  of  occupied space and to emerge on the first
vacant  piece  of  ground.   The  movement  is  instantaneous  in the given
direction.  If you are mounted or metidated then that object will accompany
as you pass.

{\bf  Pegasus}  The  pegasus  is  a  winged  horse and may be ridden by the
wizard.   The pegasus can be used to carry the wizard to safety or to mount
aerial  attacks  on the enemy.  If the pegasus is flown with a manticore or
gryphon an effective combat group can be formed.

{\bf Pit} A deep hole in the ground.  Any ground ridden creature attempting
to  cross  the  pit  will fall in and instantly die.  However, the creature
will  fill in the pit and it will no longer be effective.  Up to eight pits
can be created at considerable distance, line of sight is not required.

{\bf  Plasma  Beam}  Causes  the wizard to emit a beam of plasma which will
blast  everything  in  its  passage  as it streaks across the screen in the
chosen direction.  Simply pick a cell adjacent to the wizard to specify the
direction.  The wizard's own creatures are not immune to the beam.

{\bf  Platinum  Dragon}  The  platinum dragon or Bahamut is the king of the
good  dragons  and  is  unique.   A  summons  can  only  be granted through
meditation.   The  platinum dragon has the highest stamina of any creature,
can attack as well as a vampire, fly, and make use of a breath weapon.  The
breath  of  Bahamut  is  so  strong that it is likely to start an elemental
fire,  similar to that produced by the Fire spell.  Bahamut has exceptional
intelligence and self-healing ability.  Bahamut, of course, cannot directly
kill undeads but the fire breath can be used on undeads.

{\bf  Points}  When  competing  for  honour  against  other  wizards  it is
desirable  that  your final score reflects your ability.  This simple spell
awards  your  wizard  a  hundred  points  in  reflection  of his greatness.
Unfortunately,  the  spell  does  nothing to assist your fate in the world.
The  spell is best cast when you are stuck in a corner with no room to cast
a  more  productive  spell.   The  spell is most beneficial in limited turn
games where you are playing solely for points.

{\bf  Poison  Dagger}  A  spell  which  causes  the  chosen  target to stop
attacking  life  force and turn to more sinister means of accomplishing the
same  task.   The  chosen  target  will  instead infect during combat.  The
infection  is  very  strong  and will quickly mitigate the recovery rate of
strong  creatures  leading swiftly to death.  For humanoid races this spell
gives  the  combatant  an  infected dagger, for other creatures appropriate
action  is  taken,  ranging  from  poisonous  fangs to astral viruses.  The
dagger does 5 points of damage.

{\bf  Pool}  This  little  puddle  is much more than it appears.  Creatures
placed  next  to  the  pool will heal their wounds much faster than normal.
The  healing  will  work  on any nearby creature irrespective of the owner,
thus  great  care should be taken in the placing of the pool to ensure your
creatures derive the maximum benefit.

{\bf  Power  Wall}  A  type  of  wall  constructed  by hand and not usually
available by magic.

{\bf  Pox}  The  pox is a highly infectious and contagious disease which is
spread  by proximity.  Creatures infected with the pox will slowly wilt and
may  eventually  die.   Wizards  are  not  immune to its effects but can be
protected  with a pox shield.  All creatures can be cured of the pox by the
cure spell.  This spell allows you to infect one creature with the disease.

{\bf Pox Shield} A special shield for the wizard to guard against the pox.

{\bf Protection} A magical shield specifically for the wizard.  This should
be  applied  after  your  wizard has been magically attacked to restore the
wizards  shield.   Applying  the  shield  to  a wizard with maximum magical
protection with give additional long term protection.

{\bf  Pseudodragon} An imitation dragon which is considerably weaker than a
real  dragon,  but  which  will  strike  fear  into  opponents who have not
witnessed one before.

{\bf  Pyrohydra}  The  pyrohydra  is  an  extremely  lethal  variant of the
standard  hydra.  The pyrohydra is able to emit a burst of fire from all of
its  heads  simultaneously.   The  temperature  and destructiveness of this
breath  exceeds  that  of  all but the strongest dragons.  The range of the
breath  is  rather limited.  A pyrohydra can only be obtained by meditation
and  even  then is rather unlikely.  Most other living creatures should not
risk approaching a pyrohydra.

{\bf  Pyrotechnics}  Causes  the  wizard  to  emit a volley of rockets into
several  nearby  cells.  The spell is hard to control and often the wizards
own  creations  may  suffer  some impacts.  The spell is best used when the
wizard  is  surrounded  by  weak  opposition  creatures.   The  rockets are
particularly effective as they often explode inside their targets.

{\bf  Python}  A snake which kills its prey by constriction.  The snake is,
however, not poisonous and therefore not feared by larger creatures.

{\bf  Quench}  Extinguishes  all  fires  on the screen, irrespective of the
owner.  Only available by meditation.

{\bf  Quickshot}  Advanced  rapid-fire  training  for a wizard.  Allows the
wizard to shoot with a bow three times instead of once.

{\bf  Raise  Dead} The Raise Dead spell is a highly sought after spell.  It
enables  a  wizard  to  bring  a  corpse  back  from the dead as an undead.
Whatever  is raised comes back from the dead as if it had just been cast by
the  wizard  except  that  it will be an undead.  Don't be too quick to use
your  raise dead spell, but don't hold back on it for too long either.  The
spell has only a short range and requires line of sight.

{\bf  Range  Boost}  A  spell to improve the ranged combat abilities of any
creature  which  has  a  shooting weapon, including dragons, elves, wizards
with bows, pyrohydra and so on.  It has no effect on creatures without some
initial ability.

{\bf  Recover  Boost} This spell will increase its target's ability to heal
wounds,  provided  of  course  the recovery rate in not already at maximum.
The  spell is best cast on your wizard or a strong creature such as a hydra
or crocodile or on an undead.

{\bf Red Dragon} The red dragon is slightly superior to the green dragon in
several respects.  Like all dragons, a red dragon should not be attacked in
an  ad  hoc manner.  If you wish to kill a red dragon you are going to need
some real muscle.  The red dragon has medium magical resistance.

{\bf  Reflector}  A  special  shield for a wizard which protects the wizard
against  ranged  combat.   Any  incoming  arrow,  missile  or dragon breath
undergoes a phase shift and is redirected to the source of the attack.  The
shield is not effective against normal combat or magical attacks.

{\bf Reincarnate} Gives a creature a reincarnation sequence.  Once killed a
creature  possessing  this important property will not disappear altogether
but  will  return  to life as another, usually weaker, creature.  The giant
beetle  is  the  final state for living creatures, and the imp is the final
state for undeads.

{\bf Replicate} Make a clone of the chosen creature.  The clone will belong
to  the  same player as the original, so you best replicate one of your own
creatures.  The copy is identical in every respect.  Replicate will fail if
there  is  no  empty  space  adjacent  to  the  creature  to be duplicated.
Replicating one's own wizard causes prolonged disorientaton.

{\bf  Repulsion}  A  spell  which attempts to create space around a wizard.
Everything near the wizard is forced to move out a cell.

{\bf  Request} Sometimes a wizard does not have time to formulate a summons
for  a  particular creature.  In this spell the wizard incants at speed and
the result is unpredictable.  You may get any creature as a result.

{\bf  Restoration}  Restoration  restores  a  creature to its original mint
condition.  It also works on inanimate objects.

{\bf  Reveal}  Reveal  is  the  opposite  of  cloak.   Using  reveal  on  a
non-cloaked  creature  does  nothing,  and  is a waste of the spell.  Using
reveal  on a cloaked creature will destroy the cloak rendering the creature
visible  again.   Once revealed the creature can be attacked as per normal.
Reveal works anywhere and does not require line of sight.

{\bf  Ride}  Special  training  for  the  wizard  which enables him to ride
dragons  as  well  as  the usual creatures.  In addition the wizard will be
able to mount elephants, dinosaurs, and the bird lord.

{\bf Robot} A very strong battle machine controlled by the wizard.  A robot
can only be destroyed by beating it to sheet metal.  They are comparable to
a stone giant in abilities but much rarer and more demanding on the casting
wizard.

{\bf  Rock}  A  large  stone  that functions in a similar way to a piece of
wall.   However,  only  a single rock is available.  It can be cast on dead
creatures unlike wall.

{\bf  Roper}  Once  cast,  a  roper  moves and fights on its own accord but
remains  loyal  to  the  casting wizard.  The roper has six rope like limbs
which  it  uses  to  snare opponents and choke them to death.  The roper is
normally immobile but will occasionally teleport itself to a new location.

{\bf  Sanctuary}  Allows  a  wizard  to  recover  quicker  from  any wounds
received.  The effect remains in force for the duration of the game.

{\bf  Separation}  Breaks any alliance your wizard is currently in.  Useful
when weak wizards force you into unwanted alliances.

{\bf  Shade}  A human transformed by their own magic or dark science into a
much  more  omninous being.  The shade are rumoured to have several special
abilities, but as yet these have not been revealed.

{\bf  Shadow}  This horrible undead creature is nearly invisible and drains
an  enemie's  strength  merely  by  touch.   They are fast moving and quite
strong.   They have little magical resistance and the best way to destroy a
shadow is by a magic bolt or similar weapon.

{\bf  Shadow City} A mysterious safe haven for the wizard.  It is a home of
the dead and as such normal creatures should avoid coming in contact with a
shadow  city.   It essentially behaves as an undead.  These havens are much
rarer  than  magic  castles  or  dark  citadels.  When they collapse with a
resident  wizard  the potential benefits are high.  It is unlikely the city
will be detroyed by attack.

{\bf  Shadow Dragon} A partial reincarnation of a previously living dragon.
A  dragon  which  is  due  to  be reincarnated is likely to become a shadow
dragon.  The spell can also be cast directly, but is very rare.

{\bf  Shadow  Wood}  The shadow wood is a useful tree when they are on your
side.   The  shadow  wood  cannot  move  but it can attack any enemy living
creature  that  strays  into an adjacent square.  Unlike the magic wood the
shadow  wood  cannot  be used for meditation by a wizard.  Each cast allows
for six trees to be conjured up to twelve cells away from the wizard.

{\bf  Shape  Changer}  This  special  creature  has no single form, but can
appear  as  a  lion,  a  bear,  or a crocodile.  The abilities of the shape
changer  slightly exceed those of its constituent forms.  The shape changer
is only available by meditation.

{\bf  Shocker}  Fits  the  recipent with an electromagnetic pulse generator
which  will  output one isotropic high energy pulse per round.  Essentially
this  means  enemy  creatures  adjacent  to  this  creature  will suffer an
additional  wounding  at  the  end  of  each round.  The pulse is extremely
deadly  and  given  the  choice  an  opponent  would be better off standing
adjacent to 2 wasp nests simultaneously.

{\bf  Simulacrum}  A  weaker  form  of  the Replicate spell.  A copy of the
chosen  target  is  made.   The  copy will belong to the same player as the
original.   The  copy,  however,  will  be reasonably weak.  Simulacrumming
one's own wizard causes prolonged disorientation.

{\bf  Singular  Earthbind}  This  spell  does not require line of sight and
works  over  the  entire world.  The target of this spell loses their wings
and can no longer fly.

{\bf  Skeleton}  The skeleton is a magically animated human skeleton having
no intelligence or magical resistance and is the weakest undead.  Skeletons
are  easily  conjured  so  they  are a very common undead.  The skeleton is
useful  for  holding living creatures at bay but quickly dies when attacked
by any other type of undead.

{\bf  Sleep}  The  sleep  spell  as  the name suggests can be used to put a
creature to sleep.  Sleep does not work on wizards or on growths like gooey
blobs.   Once  asleep  a creature cannot move or attack.  It is possible to
attack sleeping creatures.  The spell can be reversed by a wake spell.  The
pair sleep-wake presents an alternative to using a subversion spell, but be
aware that any wizard can use either spell.  Sleep spells are best reserved
for  use on strong opponents.  The sleep state is identical to that induced
on creatures when a wizard dies.

{\bf  Slow} Works over the entire world and does not require line of sight.
The  target  will  only  be able to move a single cell at a time.  Does not
affect ranged combat.  Up to 3 creatures may be slowed down.

{\bf  Solar}  A powerful undead spirit which would normal only serve a good
deity,  but  under exceptional circumstances will aid a wizard.  When their
assistance  is  given  they  will  do so on their own accord.  However, the
solar will remain loyal to its casting wizard.

{\bf  Spectator}  An observant creature that has little interest in battle.
However,  prolonged  exposure  to  this  creature  can cause loss of sight,
resulting in a greatly reduced agility rating.

{\bf Spectre} The spectre is a powerful undead human having an existence in
its   own   right,  unlike  skeletons  and  zombies  which  are  maintained
continuously  by  the  wizard.   The  spectre has abilities comparable to a
wraith.

{\bf Speed} Works over the entire world and does not require line of sight.
The target will be able to move an incredible ten cells per turn.  Does not
affect ranged combat.  Up to 2 creatures may be speeded up.

{\bf Spider} A menacing hinderance to the opposition.  Although unlikely to
be fatal to any opposition the spider takes some effort to kill.

{\bf  Spriggan}  An  ugly,  dour cousin of the gnome, generally dwelling in
ruins.   They do not normally associate with wizards, and hate being in the
company  of many humanoid races.  If they become stressed they may suddenly
decide to take a nap.

{\bf Standard Wall} This is the most common type of wall in practice.  Like
the  weak wall this wall is vulnerable to magic attacks, but has a stronger
constituency  making  it less vulnerable to combat attack.  Four pieces can
be cast per spell over a limited distance.  Line of sight is required.

{\bf  Still}  Heal  the  world  form all the wounds created by earthquakes.
However, it does not prevent the possibility of future earthquakes.

{\bf Stone Giant} Capable of killing fauns, elves, horses and giant rats in
a  single  blow, the stone giant is a truly awesome creature.  Stone giants
have  been  known to kill lesser dragons and even once a pyrohydra.  What's
more the stone giant is a brilliant defender.  Once in battle, however, the
giant  must  fight  to  the death, but usually the giant will emerge as the
victor.

{\bf  Stone  Golem}  As the name suggests these unintelligent creations are
made  of  stone and are therefore very strong and can take much punishment.
The  crushing  blow  of  a  stone  golem's  fist  is  enough  to deter most
assailants.  They are, however, quite susceptible to magic attack.

{\bf  Storm}  Causes the wizard to emit a random volley of lightning bolts.
Due  to the number of bolts produced the wizard will not be able to control
where  they land or which creatures are affected.  The spell should be used
cautiously.   It  is  best  applied  when the wizard is surrounded by enemy
creatures.

{\bf  Strong  Wall} This kind of wall is only available through meditation.
Strong  wall  is  superior to weak wall and standard wall as it has inbuilt
magical  resistance,  making  it much harder to destroy.  Only three pieces
are granted per spell.

{\bf  Subversion}  Subversion  is  an attempt to convince an enemy creature
that  its  interests  might  be  better served if it had allegiance to your
wizard.   The spell might or might not succeed.  More intelligent creatures
are less likely to change sides than intelligent creatures.  You will never
be able to convince another wizard to join your side or any creature such a
wizard  might  be  riding.   Subversion  can  be  attempted on any creature
(including  sleeping creatures).  The spell requires line of sight.  If the
spell  fails  nothing  happens  to either party.  If the spell succeeds the
target  of  the  spell  immediately  swaps  to your side and will no longer
respond to the commands of its previous leader.

{\bf  Summons}  Surrounds  the wizard with a random selection of creatures.
This spell is very useful in the early stages of the game.

{\bf  Swap}  Allows  you  to  select another player with which to swap your
position.   All  creatures  you  currently control will become those of the
target  while  you  will gain control of all their creations.  Your spells,
however, will not be swapped.

{\bf  Sword of Sassenrath} Works over the entire world and does not require
line of sight.  The recipient of the Sword of Sassenrath will gain 5 points
of  combat  and  will  suddenly be able to attack undeads.  Due to the fact
that this mighty weapon was forged by the Arch-Mage Sassenrath himself (one
of the most inteligent wizards to have ever lived), the recipient will also
become considerably more intelligent.

{\bf  T-Rex} The tyrannosaurus rex is the most terrible and fearsome of the
carnivorous  dinosaurs.  Despite its size this animal is surprisingly swift
and  is  a  good  match  for any dragon.  It can kill many creatures with a
single  bite.  It is capable of killing a lion, pegasus, bear, or bird lord
within two turns.  Tyrannosaurus is best attacked with magic.

{\bf Teleport} When a strong creature is attacking your wizard or when your
wizard  is  jammed  in  a tight spot a teleport spell is useful to jump you
safely  to  a  new location.  Select the empty cell you wish to move to and
activate the spell, your wizard will magically translocate to the specified
location.

{\bf  Tempest}  A  tempest  is  an  air  elemental  which  moves at its own
volition.   Inanimate  objects are not affected by a tempest.  Any creature
falling into a tempest will be propelled to a new random location (which is
sometimes  desirable,  but  usually best avoided).  There is a small chance
that the transported creature will not survive the ordeal.

{\bf  The  Exorcist}  The  exorcist is very useful for purging the world of
enemy undead creatures.

{\bf  Thundermare}  A  rare  enchantment  over  the  standard  horse.  This
creature  is  very  rarely  available  as  a free spell, instead most often
occurring  when a wizard riding a horse successfully kills another creature
with  the horse.  The thundermare has numerous advantages over the standard
horse,  although  its base strength is only slightly higher than the normal
horse.  Its biggest asset is its advanced combat training.  The thundermare
is  capable  of  lashing  out with all hooves simultaneously and still land
back  on  them  afterwards.  The thundermare represents an ideal safe haven
for  the wizard.  It has extreme magic resistance and will probably only be
killed in direct combat, if at all.

{\bf  Torment}  Allows  your  wizard  to influence the strength of creature
casts  made by all the other wizards.  When another wizard casts a creature
its  life  force will be only 4 points, making it easy to kill.  If you are
lucky  the  effect will hold for several creature casts of each wizard, but
more  likely  it  will  work  once  for  each  wizard.  It has no effect on
independents or generated creatures.

{\bf  Touch  of  God}  A  very powerful spell which can turn the weakest of
creatures  into an extremely intelligent and powerful being.  The spell can
also be cast on the wizard.

{\bf Triple} This is a highly sought after spell which gives the wizard the
ability  to  make  three attempts with each subsequent spell, for a limited
time.   It  does  not  work  with  a  few taxing spells like magic wood and
justice.  It is most valuable near the start of the game when building your
forces.

{\bf  Troll}  The  troll  has an uncanny ability for being able to heal its
wounds  very  quickly but is a rather stupid creature.  Trolls enjoy a good
fight,  but are more suitable as accomplices to a stronger creature such as
a stone giant in a major offensive.

{\bf  Turmoil}  Turmoil  affects  every square in the world.  Every cell is
randomly  transposed  to  a new location.  The result is very unpredictable
and  may  or may not improve your situation.  Often useful when attacking a
well  fortified  opponent.  This spell is only available through meditation
and is very rare.

{\bf  Turns} Are you tired of expending a large amount of energy struggling
in what seems to be a never ending battle.  This spell could be answer.  If
the  number  of  turns  remaining  exceeds  ten  then  this spell will give
everyone  a major hurry up by cutting the number of remaining turns to just
ten.  However, the spell is bipolar and if the number of turns remaining is
less than ten then an additional ten turns will be granted.

{\bf Ubiquiscope} The Ubiquiscope is equivalent to a Crystal Ball + a Magic
Wand  combined.   A  highly desirable spell which removes the need for your
wizard  to  have  line  of  sight  in  order  to  cast  a  spell.  With the
Ubiquiscope,  your  wizard  can "see" everywhere.  Range restrictions still
apply as usual, though each Ubiquiscope will add +1 to your casting range.

{\bf  Uncertainty}  Allows  the  wizard  to harness the energy provided for
under  the  quantum uncertainty principle.  For several turns after casting
this  spell it is likely that creatures will just pop into existence on the
side of the casting wizard.

{\bf  Unicorn}  Unicorns  can be ridden by a wizard but have little else to
recommend  for  them.   While  a unicorn's horn can inflict a serious wound
most unicorns are not particularly skilled in combat.

{\bf  Vampire}  The  vampire  is  amongst the most feared of all creatures,
second  only  to  the  demon.   The  vampire  has exceptional intelligence,
excellent  fighting  ability  and  a  good  magical  resistance  as well as
endurance.   Vampires  are  known  to  have  killed  golden dragons in five
attacks.   The  chances  of a vampire being killed in any direct combat are
minimal  unless  the  vampire  is attacked by other vampires or by a demon.
Usually a vampire is only destroyed by magical means.

{\bf  Vanish}  Cause  the  chosen  target to move to the astral plane.  The
object  will instantly disappear and not return for some rounds.  The spell
can  be  applied to anything including generators and enemy wizards or even
yourself.

{\bf  Vengeance}  Vengeance  is used to seek vengeance against an opponent.
Vegeance  cannot  be  applied  to inanimate objects but may be used against
anything  else.   The  chosen  target suffers three points of damage to its
magic resistance.  Line of sight is not required.

{\bf  Violet Fungi} Once cast a violet fungi no longer responds to commands
from  its  creator,  although it will not harm its creator in any way.  The
violet  fungi  releases  toxic  spores  which  reduce the recovery rate and
stamina  of  nearby opponents.  Occasionally, a fungi may sprout additional
fungi.

{\bf  Virtue}  A  vapourous  blast of multichromatic light resulting in the
death  of  many white coloured species.  The caster's own creatures are not
immune.  The following species are immediately vapourized:  air elementals,
arctic  wolves,  bats,  derros,  eyes, falcons, fauns, fire bats, gryphons,
pegasii,  rats,  skeletons,  stone  giants, stone golems, trolls, and white
dragons.

{\bf Vitality} A strong encouragement to your creations.  This spells gives
them peace of mind which means that their wounds will heal at a faster rate
than normal.  Its effect is cumulative.

{\bf  Vodka}  This  spell  causes bottles of Vodka to appear nearby all the
creatures  controlled by the same wizard as your target.  All the creatures
will  be very happy about receiving this free gift and will merrily chug it
all  down.   The  end  result  is  that  they will all be drunk for 2 turns
cumulative and their combat ability will be effectively cut in half.

{\bf  Volcano} Unleashes powerful tectonic forces resulting in the creation
of  a small volcano.  Any creature nearby will become more agile.  However,
the volcano is inherently unstable and can explode without warning.

{\bf Vortex} A roaming temporal anomaly that moves slowly around the board.
Creatures  caught  up in the vortex will become temporally distorted.  They
are typically propelled several rounds into the future.

{\bf  Vulture}  The vulture is easily cast and unlike most creatures can be
cast  up  to  two  squares  away from the wizard, provided there is line of
sight.  The vulture easily escapes from conflict and has good range.

{\bf  Wake} The wake spell is used to reverse the effect of the sleep spell
or  to wake creatures which went to sleep when a wizard died.  Once woken a
creature  will  be  under  the  allegiance  of the wizard who cast the wake
spell, irrespective of how the creature was put to sleep.  Quickly make use
of this spell when the chance arises as sleeping creatures tend to decay.

{\bf  Wasp Nest} A wasp nest is pretty self-explanatory, any enemy creature
approaching  a  wasp  nest will be automatically attacked.  Once you cast a
wasp  nest you can forget about it, since you cannot move it and it attacks
automatically.   It  will  attack all adjacent squares simultaneously.  The
wasps are friendly towards the creations of its owner.

{\bf  Water  Elemental}  A dangerous creature closely resembling an undead.
Upon its death it decays into a flood which can expand to cover the world.

{\bf  Weak  Wall}  Despite its name, the weak wall is a formidable obstacle
for  an  opponent  to  overcome.   It  is  extremely useful for slowing the
passage  of  fire,  or  blocking  the  growth of gooey blob.  The name only
implies  that  this wall is weaker than standard wall or strong wall.  Most
magical attacks will eliminate a weak wall.  The wall can also be destroyed
by  combat  but  this will usually take several rounds.  For each weak wall
spell six pieces of wall may be cast anywhere on the screen.

{\bf Web} Allows the wizard to magically create a few pieces of spider web.
Such  web  is  useful  for  blocking  weak  opponents,  but will not stop a
determined opponent.

{\bf  White  Dragon}  White  dragons  normally  dwell only in cold regions.
Meditation  may  allow  a  wizard  to  call upon a white dragon.  The white
dragon  is  very  strong  but  slower  than  other dragons due to its lower
metabolic  rate.  The white dragon can freeze other species to death with a
breath  weapon  which  is more effective than the breath weapons of the red
and green dragons.

{\bf  Wight}  The  wight is an evil undead and falls into the middle of the
undeads in terms of overall ability.  They can be beaten by wraiths, ghosts
and vampires but can deal to skeletons and zombies.

{\bf Wizard} A pseudowizard which often confuses other players as to who is
the  real wizard.  The fake wizard is unable to cast spells but can partake
in  normal  combat.   The  spell  is only available by meditation, although
generators will produce fake wizards frequently.

{\bf  Wizard  Wings}  Wizard wings gives your wizard the ability to fly and
will  set  his  movement  to  6.   The spell will only work on your wizard.
Provided the wizard is not engaged or dismounting, flight with the range of
an  eagle  will  be  possible.   Wizard  Wings  are particularly useful for
hopping between magic wood trees.

{\bf Wolverine} A ferocious mutant wolf that loves killing and is very good
at  it.   Wolverines normally shun wizards and can only be obtained through
meditation.  Apart from its superior fighting ability the wolverine is much
the  same as the dire wolf.  They loathe dire wolves which they consider as
unwanted competition for food.

{\bf  Wood  Elf} Like all elves the wood elf has a pure heart giving them a
high  magical  resistance  (particularly  from  the darker kinds of magic).
Wood  elves  are physically weak and should best avoid direct combat.  Wood
elves are highly skilled with a bow.  Every wood elf comes complete with an
ash  bow  and  a  full  quiver  of  arrows.  The ability of the wood elf is
slightly exceeded by their rarer cousin the gray elf.

{\bf  Wraith} The wraith is closely related to the wight but is superior in
several  respects.   Wraiths  have  exceptional  intelligence and excellent
magical  resistance.   The wraith can outrun most lesser undeads but rarely
has the need to, since the wraith is a proficient killer.

{\bf  X-Ray}  Blast an object with a high dosage of X-rays.  Weak creatures
will  die instantly, other creatures might survive a few blasts but will be
weakened.  Line of sight is required.  Can be used on inanimate objects.

{\bf Zombie} The zombie is a magically animated corpse and is only slightly
more  sophisticated than a skeleton.  Like the skeleton, the zombie is very
effective  at blocking the progress of land-ridden living creatures.  Other
undeads, however, can quickly dispose of the zombie.


\end {document}
