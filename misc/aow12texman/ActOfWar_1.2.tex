\documentstyle[titlepage,12pt]{article}

%% Setup page, etc.
\textwidth16cm
\textheight 640 pt
\oddsidemargin0mm 
\evensidemargin0mm
\topmargin0mm
\parskip\medskipamount
\pagestyle{myheadings}
\markboth{}{\sl Act of War User's Manual}

%% How about a nice logo?
\font\actb = cmss17 scaled\magstep5
\font\acts = cmss17 scaled\magstep3
\newcommand{\biglogo}{
\mbox{\actb Act}\mbox{\acts\raisebox{1mm}{$\;\;$of$\;\;$}}\mbox{\actb War}}

%% Let's have proper quotes...
\newcommand{\quo}[1]{\lq\lq{#1}\/"}

%% \key defines a "keyboard shortcut: xxx Action neeed: xxx line"
\newcommand{\key}[2]%\shortc{key shortcut}{MP needed}
{\hfill\parbox{4.5cm}{Keyboard: #1}\parbox{4 cm}{Action needed:\hfil#2\hfil}}
\pagestyle{empty}

\begin{document}


\begin{titlepage}
\title{\biglogo}
\author{Version 1.2\\
\mbox{}\\
A Game by Dave Smith
\thanks{Manual \LaTeX'ed 13--Dec--92 by Thomas B\"atzler, BAETZLER{\tt @}IRAV17.IRA.UKA.DE}
\\csc361{\tt @}cent1.lancs.ac.uk}

\date{September 1992}
\end{titlepage}
\maketitle

\newpage
\pagenumbering{roman}

\tableofcontents

\newpage

\pagestyle{myheadings}
\setcounter{page}{1}
\pagenumbering{arabic}

\section{System Requirements}

Act Of War works under both Workbench 1.3 and 2.0, in both NTSC and PAL,
and requires 1 meg of memory as a minimum. The game may need more than a
meg to run it from a hard drive.

If you have a meg and the game refuses to run, try disconnecting or turning
off any external drives or hard drives. This frees up more memory.

It should work fine on accelerated systems, if a little fast in some places!
I've got some \quo{slowdown} routines in it, but as I've only got a 68000 I
haven't been able to test it.

As Act Of War was written with AMOS, it multitasks - use Left-Amiga A to
switch between the game and Workbench.


\section{Introduction}

Act of War, or AoW as I've just decided to call it as it involves less
typing, is a strategy game in the tradition of Laser Squad, Breach 2 (or so
I'm told) and, if anyone can remember that far back, Rebelstar.

If you've never played any of those games, think of a childish fascination
for big guns married to a fair amount of strategy and you won't be far
wrong.

\subsection{So what makes AoW different?}

The game is played over a series of missions, three of which are included
with the program. More missions are being worked on as you read this, by
myself and others. And if you don't like these missions, there's no reason
for you not to create your own!

On payment of a small shareware fee (more information later on) you'll get
the mission designer that I use, which makes mission creation very simple.
Everything from the map, to the weapons, to the sound effects can be
changed.

\subsection[Notable game features]{Noteable game features include:}

\begin{itemize}
\item 1 or 2 players
\item Sampled sound
\item Completely extendable!
\item 3 difficulty levels
\item Optional line-of-sight feature
\item Squad arming sequence
\item Explosive scenery
\item Keyboard and / or mouse controlled
\item Halfway-decent user interface
\item Revolutionary new ICASS AI system (see later)
\end{itemize}

\subsection{New features for Version 1.2:}

\begin{itemize}
\item Fire (as in flames)
\item Units can return fire even if it's not their turn, under certain conditions
\item Character/cursor movement can now be carried out using the mouse
\item Weapons now have a range
\item Four new types of weapons:
    \begin{itemize}
    \item Scatter
    \item Shrapnel
    \item Incendiary
    \item Multihit
    \end{itemize}
\item Landmines (and scanners to find them)
\item Reinforcements
\end{itemize}

The new features are described in detail later in this document. If
you've played an earlier version of AoW and don't want to plough through
this whole file again, the file \quo{update.doc} also contains the new
descriptions.

The three missions included in this release are as follows:

\begin{description}
\item[Informant] --- An ex-director of your shady corporation is ready to
         spill your industrial secrets to a rival company. There's only one
         way to stop him - break into his sprawling mansion and take him
         out! (Obviously, negotiation comes a long way down the list with
         these people).

\item[Escape] --- Taken prisoner by a corrupt Planetary Security Force for your
         part in the above escapade, you and two companions have been
         jailed in an orbital maximum-security prison. Initially armed with
         only a little smuggled-in plastic explosive, you must break out of
         your cell, somehow obtain a weapon, free your comrades and escape
         to a passing PSF shuttle to secure your freedom.

\item[Terminator!] --- Soon after you make it back to your base after escaping
         captivity, a Terminator droid from a rival corporation stages
         an assault! Backed up by a squad of combat droids, it will attempt
         to kill you all $\ldots$
\end{description}


\section{Installation}

AoW can be installed on a hard drive - just make sure that the files:

              diskfont.library           mathtrans.library

are in the libs: drawer, and that you've copied the \quo{Spacey} font to the
fonts: drawer. The directory that you put the game in must have the name
\quo{px:} assigned to it. In other words, if the game is in
\quo{dh1:Games/AoW}, you should put:

                       assign px: dh1:Games/AoW

in your startup-sequence.

The mission directories should be in the same dir as the game, ie.
\quo{px:}.  The default missions also require that the \quo{Pics} drawer
and the \quo{Snd} drawer, and their contents, are present in \quo{px:}
also.

The game is much nicer installed on a hard drive (isn't everything?). As
it is so extendable, a fair amount of disk access is inevitable. However,
there is no disk access during a game so the actual gameplay speed is
identical on both floppy and hard drive systems.

To summarise then: the necessary files are as follows.

\hfil\begin{tabular}{l}
         libs:diskfont.library\\
\hspace*{1cm} mathtrans.library\\
         fonts:Spacey/7\\
\hspace*{1cm}Spacey.font\\
         px:Pics/GameTiles\\
\hspace*{1cm} EscapeTiles\\
\hspace*{1cm} GameBack.iff\\
\hspace*{1cm} WeaponBack.iff\\
\hspace*{1cm} Weapons.1\\
         Snd/Samples.1\\
         Act\_Of\_War\\
         Informant/Everything\\
         Escape/Everything\\
         Terminator!/Everything\\
\end{tabular}\hfil

\pagebreak

\section{Gameplay}

I know, I know, you just want to know how to play!

\subsection{The front end}

\begin{itemize}

\item The first thing you have to do is load a mission in.  Select \quo{Load
    Mission} from the menu. The disk will grind a bit, and a list of
    the available missions will come up. Choose the one you want,
    again using the mouse.

\item Select the difficulty level you want. Easy gives you a lot of cash for
    the arming sequence (see below). Medium gives you less cash and
    introduces the line-of-sight feature, where you can only see enemy
    units that are in your line of sight. Hard keeps line-of-sight and
    gives you the least cash.

\item Choose a 1 or 2 player game. The AI is not brilliant, but I think it's
    good enough to be called \quo{not bad}. It even has its own acronym, which
    is obviously essential for any program these days. So, when you play
    a 1-player game of AoW, you are experiencing ICASS - the Intellectually
    Challenged Artificial Stupidity System (tm).
\end{itemize}

    Now, don't you feel proud?


\subsection{Arming}

\begin{itemize}

\item  Next comes the arming sequence. A total amount of cash is shown,
    and you can either accept the default weapons (use the \quo{Next Unit}
    icon, which looks like a tick, to see each unit's weapons) or choose
    to buy your own. To keep the defaults, click the \quo{Quit} icon; to buy
    weapons, click either of the two \quo{Buy} buttons.

\item The squad's default weapons are usually round about what you could
    buy at the Medium level of difficulty. The amount of cash you get at
    each level is as follows:

\hfil\begin{tabular}{lcl}
Easy   &  --- & 250 per unit\\
Medium &  --- & 200 per unit\\
Hard   &  --- & 175 per unit\\
\end{tabular}\hfil

\item  If you do buy new weapons, the first button is \quo{Buy Weapon}, the next
    is \quo{Buy Ammo}, the two arrow icons let you see the weapons and the
    \quo{Next Unit} button is used when you've finished arming that unit. You
    can always go back to it, until you quit the arming.

\item  The cash shown is a total for your whole squad, try to divide weapons
    fairly equally. It's always better to have, say, four men with Pulse
    rifles rather than one man with a Plasma Cannon and tons of ammo.

\item  If you buy something, then decide you don't want it, you can sell it
    back at no loss by clicking on the object's name in the inventory.

\item  Click on the Quit button (bottom right) when you've finished.

\end{itemize}


\subsection{The game}

\begin{itemize}
\item The screen is divided into four windows:
\begin{enumerate}
    \item The largest window, at the top left, is the view window. This
         is where the action happens.
    \item Directly below this is the message window. Always keep half an
         eye on this for reports and updates.
    \item At the top right is the status window. Here you can see the stats
         of the current unit.
    \item The final window at the bottom right is the mini-map window.
         Apart from giving you an idea of the overall shape of the map,
         you can keep track of your units here via the flashing dots
         which show their positions.
\end{enumerate}

\item Player 1 always moves first. You begin the game fully armed and ready.

\item There are several ways of moving around the map.  You can use the
    cursor keys, the joystick or the small direction buttons with the mouse to
    move the cursor around.  You can click with the left mouse button on the
    view window to place the cursor directly onto a square.  You can click on
    the mini-map to jump directly to another area.

\item To select a character, press the joystick fire button or the space key
    when the cursor is over a unit. You can also click the right button
    on a unit to select it. You are now in move mode, where believe it or
    not you get to move the unit around.

\item To move a unit, use either the cursor keys, the keypad or the joystick.
    The keypad allows movement in 8 directions, using the keys:

\label{keypad}\hfil\begin{tabular}{ccc}
                               7 & 8 & 9\\
                               4 &   & 6\\
                               1 & 2 & 3\\
\end{tabular}\hfil

     If you have an A600 without the keypad, you can use the normal
     number keys with the same effect.

\item To deselect a character, use the space key, the joystick fire button or
    the right mouse button at any time.

\item If you are playing on Easy level, you'll be able to see all the
    enemy units as you look around the map. This takes a lot of the fun
    out of the game, so the default difficulty level is Medium. Here,
    you can only see an enemy if they are in a direct line-of-sight. This
    allows them to hide around corners and jump out at you, wait behind
    doors etc. Remember, you can see through windows.

\item If you find a scenario too tough on Medium, or you don't like the
    hidden movement, go to Easy. Otherwise, I'd recommend you stay on
    Medium or Hard.

\end{itemize}

So now you know how to move around the map, and select characters (by the
way - I use the words \quo{character} and \quo{unit} interchangeably, they
both mean the same).  Now things get a little bit more complicated!

Nearly every action in the game can be carried out by the mouse or the
keyboard. If you prefer to use a keyboard, you can play the whole game
without having to take your hands away from it. However, I prefer a mix of
the mouse and keyboard. It's up to you though.

You'll have noticed (assuming you've actually played the game yet) six
buttons by the mini-map window. I'll go through these and any sub-menus
that they lead to.

First, a word about actions. Everything you do in the game requires a
certain amount of actions to carry out, and you have a set number of
actions per unit, per turn. For instance, moving across a normal square uses
2 actions, but moving over rubble and debris takes 3. Picking something up
takes 5 actions, teleporting takes 20. As I explain each option, the
number of actions it needs is shown alongside.

If a character's health is reduced to 1/3 of his (or her!) maximum, they
are said to be injured and their actions are reduced by half. You can
tell when a character is injured - their health and actions show up red in
the status window.

\begin{description}

\item[MOVE] \key{SPACE\footnote{Only valid when in FIRE mode}}{---}

The first button, with the four outwardly-pointing arrows, is the MOVE
button. When a unit is in FIRE mode, this button exits and returns it to the
default MOVE mode.


\item[FIRE] \key{F}{---}

The gunsight icon begins FIRE mode if a unit is selected. This changes the
cursor to a gunsight, and brings up a sub-menu in the mini-map window (all
the sub-menus appear here).

The cursor can be moved in the same way that characters are moved around.
When you've moved it over a target, select a type of shot from the menu,
using the mouse, or one of the keyboard short-cuts:

\begin{description}
         \item[Aimed shot] \key{A}{varies}

         Fires an aimed shot towards the gunsight. The number of
         action points needed is shown.

         \item[Fast shot] \key{F}{varies}

         Fires a fast shot towards the gunsight. Actions needed vary
         according to the weapon you're using.

         A Fast shot requires only 2/3 of the action points needed
         for an Aimed shot, but has only 2/3 the chance of being
         accurate. This doesn't sound very inaccurate, but in practice
         you'll find it's more than enough to miss rather a lot!

         \item[Throw Grenade] \key{G}{6}

         If a character has a grenade, throw one toward the cursor.
         The cost is always 6 actions.

         Grenades have several special properties. Naturally, they are
         explosive, but where a shot will be stopped by a window, a
         grenade will continue through it. Where a shot will hit a desk
         or other piece of furniture, a grenade can be chucked over it.
         Where a shot will hit a character between you and your target,
         a grenade can be thrown past them. Grenades will always head
         towards where you threw them - there is no such thing as an
         off-target grenade.

         There is a downside to this. Grenades are not the Ultimate
         Weapon! A grenade can only be thrown a finite distance, unlike
         the other weapons which have no maximum range. A fairly strong
         character, such as a Battle Droid, can throw one maybe 7
         squares (think of this as about 14 metres). An average human
         can manage about 5-6 squares. They are also expensive for a one
         shot weapon.

         \item[Load Weapon] \key{L}{varies}

         Assuming you have any ammo, this loads your current weapon.
         Actions needed vary according to the weapon you have.

\end{description}
\end{description}

Weapons can have several special attributes. These are Explosive, Scatter,
Incendiary, Shrapnel and Multihit.

Explosive weapons kill outright, unless the target is incredibly tough. They
also affect the environment around the target, eg. walls may be knocked
down, windows smashed, other units injured, volatile substances ignited
etc. They are very powerful, and you'll probably find that they're priced
accordingly!

Shrapnel weapons are a subset of explosive weapons, which not only explode
but also throw out shrapnel over a certain area. Characters not in the
actual area of explosion can therefore still be injured.

Incendiary weapons set fire to their targets without actually exploding. The
target does not have to be combustible to catch fire - weapons such as
napalm will even burn on water. A flamethrower not only sets fire to its
target square, but also all the combustible squares leading to it.

Scatter weapons spread out when fired, allowing several target squares to be
hit with just one shot. A shotgun is a good example of a low-powered scatter
weapon, whereas a triple-grenade launcher is also explosive, and can be
unfeasibly dangerous.

Finally, multihit weapons are very powerful. Like Scatter weapons, they can
hit several targets with one shot, but unlike Scatter only one line-of-fire
is used. In other words, the projectile fired carries on in a direction
until its power has been exhausted. With an extremely powerful weapon such
as a Particle Cannon, a unit can actually shoot through a wall and hit an
enemy standing on the other side!

Non-explosive weapons, the majority, simply damage characters. Armour worn
by a character affects how much damage they sustain. Beware of firing any
type of weapon near things like oil drums, petrol tanks, ammo crates etc.
Even non-explosive weapons set them off! Windows, too, can be smashed by
any type of weapon.

You have to be very careful when shooting at the enemy, as under certain
conditions they can return fire immediately. If a unit ends a turn with 50\%
or over of its actions unused, and has a skill of 8 or more, it can return
fire when shot. Obviously, player units can do this too, it's not restricted
to the computer.

Hand-to-hand weapons are also available, though the only one available in
the three included scenarios is the Laser Knife. These are used purely for
hand-to-hand combat.

This type of combat is initiated during a character's movement, by walking
into an enemy unit. The attacker always has the advantage, but if the enemy
is strong and skilful, it is possible for the aggressor to be injured rather
than his target. You always have to be careful during hand-to-hand combat!
If possible, make sure your enemy is less suited to it than you.

\begin{description}

\item[CENTRE VIEW] \key{C}{---}

This button centres the view screen on the square currently occupied by the
cursor. You can recognise it by its four inwardly-pointing arrows.


\item[OPTIONS MENU]

Brings up a sub-menu in the mini-map screen.

Which menu you get depends on whether you have a character currently
selected or not. If you don't, you get the standard menu:

\begin{description}

         \item[Next Unit] \key{N}{---}

         Moves the view to the next selectable unit.

         \item[Success Rate] \key{P}{---}

         Each game is won on the destruction of all enemy units, or
         the accumulation of success points. You are always told when
         your success rating goes up, and in the documentation for
         each individual mission you should be told exactly what gets
         your success up. It can be reaching a particular place, or
         blowing it up, or killing a particular enemy. The game is won
         when you or your opponent has a 100\% (or over) success rating.

         Anyway! This option shows your current rate in the message
         window.

         \pagebreak

         \item[Turns Left] \key{T}{---}

         Some missions have a time limit. Selecting this option
         shows you how long you've got left, or tells you which turn you
         are in, followed by \quo{Unlimited time available}.

         It's useful to know which turn you are currently in, as in
         some missions you may recieve reinforcements at a certain time.

         \item[Save Game] \key{Shift-S}{---}

         This brings up a file selector. Choose the name you want
         to save the game as, and press RETURN or choose OK with the
         mouse. Each save game is approximately 10-12k long.

         A few notes on the AMOS file requestor: at first glance, it
         appears to have no buttons for \quo{Parent} or to bring up a list
         of active drives and assigns. Actually, it can do both things,
         in its own cute but slightly eccentric way.

         To get to the parent of a directory, click on the tiny circle
         icon in the top-leftish corner, above the two microscopic
         arrows. To get a list of active drives, use the right mouse
         button.

         \item[Load Game] \key{Shift-L}{---}

         Loads a game, via the same requestor as above.

         \item[Quit Game] \key{Shift-Q}{---}

         Quits a game. There's no \quo{Sure?} type warning, so make sure
         you really want to quit before you choose this!

\end{description}


The other menu is called up when you have a character selected. It is from
this menu that most of the interesting actions in the game can be accessed
from.

\begin{description}

         \item[Inventory] \key{I}{---}

         Brings up a display of the items in your inventory. Space or
         selecting CANCEL quits (note: you can't select CANCEL if you
         called the inventory up via the keyboard. If you start an option
         with the keyboard, you can't use the mouse with it. This applies
         to all the options).

         \item[Success Rate] \key{P}{---}

         Shows your current success rate in the message window.

         \item[Pick Up] \key{G}{4}

         Pick up something from the ground.

         When you've shot and killed an enemy, you can, if you like,
         stand over his body and unashamedly loot it. This is often
         an essential tactic! All's fair in love and war (although it's
         often considered rude if you kill someone and loot the corpse
         if love is concerned).

         Objects can also be lying around - they are usually well
         marked on the map by a chest, locker, load of objects scattered
         around etc. Just seeing what is in a square costs nothing. It's
         only if you actually get something that your actions are lowered.

         \item[Drop] \key{D}{4}

         Lets you drop an object from your inventory.

         Note: any weapon you drop will have its ammo reduced to 0.
         This is NOT a bug, it is a \quo{feature}. Although I admit, it's
         a pretty bloody stupid feature.

         \item[Load Weapon] \key{L}{varies}

         Loads your weapon, if you have any spare ammo.

         \item[Change Weapon] \key{W}{5}

         If you have a weapon, this swaps your current weapon for one
         in your inventory (you choose it, of course). If you have no
         weapon, you choose one to use.

         Note: grenades and plastic explosives don't need to be held as
         weapons to be used, they just need to be in your inventory.

         \item[Next]

         Flips to the next option screen.

         \item[Open Door] \key{O}{2}

         Opens a door adjacent to you. If there is more than one, you
         can choose the correct one by clicking on the arrow pointing
         to the door.

         If you have to click on an arrow, you can use the keyboard
         instead. (cf p.~\pageref{keypad}.)

         This technique works with all the menus with arrows to click
         on, such as Shut Door and Use Explosive (see later).

         If you're going to walk through a door, there's no need to open
         it first. Just walk into it, it'll open for you.

         \item[Shut Door] \key{S}{2}

         Operates identically to \quo{Open Door}, except in one vital
         respect (it closes them).

         \item[Teleport] \key{T}{20}

         If you stand on a teleport, this option operates it. It uses
         a large amount of actions, as it's a bit draining to be split
         up into your component molecules and stuck back together
         several hundred metres away. Or so I've been told.

         \item[Use Medikit] \key{M}{8}

         Medikits restore a unit to full health. If a character is
         injured, the injury is healed also.

         Of course, you need to possess a medikit to be able to use one.


         \item[Place Mine] \key{---}{8}

         If you have a mine, clicking on this option allows you to
         choose a place to lay it. Mines can only be placed on floor
         squares, and once placed they are invisible to both sides.
 
         A unit which steps on a mine is killed, and shrapnel from the
         blast may injure nearby characters.

         \pagebreak

         \item[Use Scanner] \key{---}{6}

         Invisible mines with no way of discovering them would obviously
         be a little unfair, so scanners can be purchased in the arming
         sequence. 

         Scanners show up any mines in the immediate area, ie. the area
         on-screen. A flashing disc in a square shows that a mine has
         been detected there. There are two ways to deal with a detected
         mine: walk over it, or avoid it. I would recommend the latter.

         \item[Prime Explosive] \key{X}{6}

         If you have any plastic explosive, this option allows you to
         set the timer on it. 1 turn means detonation at the end of
         this turn, 2 means at the end of the enemies' turn, 3 means at
         the end of your next turn, 4 means at the end of the enemies'
         next turn but one etc.

         Click on Increase or use the \quo{+} key to increase, Decrease or
         the \quo{--} key to reduce the time. The SPACE key or selecting
         OK finishes.

         Make sure you have enough actions to get rid of the explosive
         once you've set it!

         \item[Use Explosive] \key{U}{6}

         Plants one bit of explosive you have primed. Click on an arrow
         to place it. Placing more explosive in one area does not
         increase the size of the explosion.

         \item[Previous]

         Click on this to go back to the first option screen.

\end{description}


\item{REDRAW MAP}

After you've finished with the Options menu, you may want to see the map
again. This button carries this out.


\item{END TURN}

The final button ends your turn, allowing the computer or Player 2 to have
their go.

\end{description}

That's the basic gameplay. There's room for a decent amount of strategy in
there, and of course lots of explosions. I mean, let's get our priorities
right.


\section{ShareWare}

As it says when you load the game, Act Of War is a SHAREWARE product,
though I bet you don't need it shouted at you.

AoW has taken me nearly four months of practically full-time effort to
write. Assuming a professional programmer is paid \pounds 10K a year, I make that
\pounds 3 grand you lot owe me!

Don't worry, just joking.

However, if you like the game, I would like a minimum of \pounds 5, which isn't
too much really (better than \pounds 3000 anyway). If you {\bf really} like the game,
I'd very much appreciate it if you sent more. If you think it's commercial
quality, you could send \pounds 10 or \pounds 15, or even more if you've got more money
than sense. Of course, you get benefits yourself!

A \quo{donation} (polite word for payment) of \pounds 5 will get you a nice letter and
a disk containing the mission editor, the latest version of the program and
any extra missions I've got. \pounds 10 will get you a {\bf very} nice letter, the
mission editor disk, and the source code of the program (all 275K of it)
on a second disk. If you send me more than \pounds 10, you'll get an unspeakably
nice letter, all the above, and probably Christmas cards for life as well.
In addition, I'll nominate you for \quo{Best Supporting Actor} at next year's
Academy Awards, and make sure your house is looked after when you go
abroad. If you know what I mean.

I'd like the money in either:

\begin{itemize}
\item   A British cheque or postal order, or
\item   UK Pounds Sterling (cash) at your own risk.
\end{itemize}

When I tried to cash some US cheques recently, I was told the minimum
charge would be \pounds 10. So, it has to be UK currency I'm afraid, unless you
make me an offer I can't refuse (like a \$500 cheque, or your DAT player, or
your sister).

Now all you need to know is where to send the money!

From now (September 1992) until June 1993, with gaps for Christmas and
Easter, you can reach me at the following address:

\hfil\begin{tabular}{c}
                           98 Thornton Road,\\
                           Morecambe,\\
                           England\\
                           LA4 5PJ\\
\end{tabular}\hfil

During the gaps (approximately December 15th - January 15th, and March 15th
- April 15th) and after June 1993, you can reach me at:

\hfil\begin{tabular}{c}
                           4 Cleveland View,\\
                           South Bents,\\
                           Sunderland\\
                           England\\
                           SR6 8AP\\
\end{tabular}\hfil


\section{Updates}

This is version 1.2 of the program (if it was a Mac program it would be
version 1.0.2). I've tested it as much as possible, to the extent that I'm
sick to death of it, so it should be as bug free as possible. If you
find any bugs, or have any suggestions for a future version, please let
me know when you register.

There's no reason why I can't do a V1.3 of the game, so any suggestions
will be gratefully received.

The most fun part of the game is of course mission designing, so I'm working
on a few more. I'll distribute these in a couple of ways:

\begin{itemize}

\item Upload them to FTP sites.

        If you don't know what an FTP site is, you probably can't
        reach one! Basically, an FTP site is an on-line software
        library, usually allowing anonymous access. They can be reached
        over the Internet/JANET, so if you're at University you can
        probably use the mainframe to use them.

        My preferred sites are:

               amiga.physik.unizh.ch   (130.60.80.80)

               wuarchive.wustl.edu     (I forget the number)

\item When I've got another diskful, I'll send it to UK PD libraries.

\end{itemize}

Hopefully, soon other people will be doing missions too, so if the game
takes off you should be seeing a lot of missions eventually!

Missions that I and others are working on at the minute:

\begin{itemize}
\item Firefight - A fight in an ammo factory
\item Arena     - Combat in an alien arena
\item Meltdown  - Destroy a reactor, and escape before it blows
\item Supergun  - Break into a complex and steal a new experimental
                  weapon, then use it to get out
\item Zoo       - Alien exhibits have escaped!
\item Haunted   - Destroy a mansion populated by werewolves, vampires,
                  ghouls... lots of silver bullets and stakes here!
\item Disney    - You know as much as I do!
\item Fantasy   - Wizards, elves, orcs etc.
\end{itemize}

There's only one way to make sure the game takes off... register!

As for my next project... does anyone remember a game called \quo{Krakatoa} on
the Spectrum? This was released in about 1983 or 84, it was reviewed in the
first issue of Crash magazine anyway. It was stunning! You were this
helicopter pilot, who had to protect a village and an oil tanker from a huge
volcano. Also, there was an enemy who launched rocket attacks every so on.

You could do all these brilliant things like bomb the oil tanker until some
survivors jumped into a lifeboat, rescue them from the lifeboat, then fly
them to the volcano and drop them in! This set off the volcano, which
erupted all over this innocent little village, sending people running from
their homes, ready to be rescued, or bombed, or landed on, or shot, or
drowned, or dropped from a great height, or lowered into a volcano.

It really was something else. Anyway, I'm planning to do a version of this.
I don't know who the original authors were, but if you're by any chance
reading this, let me know. And if you want to see a game like that on the
Amiga, register for AoW so I have an incentive!


\section{Acknowledgements}

It's impossible to do something like this entirely on your own, and so I've
got a few people to thank.

First, my girlfriend Louise for putting up with endless boring monologues on
the difficulties of AI programming and the benefits of so many enormous guns
when faced with a rampant Battle Droid.

Gary Dietachmayer helped an enormous amount with the AI, without his
brilliant ideas this may well have ended up two-player only. Oh, and he
was incredibly patient with my useless terminal and its failed attempts at
uploading files (not to mention the frequent mis-spelling of his name).

Martyn Brown, of Team 17 Software, granted permission for me to use several
samples from the brilliant game \quo{Alien Breed}.  Listen out for that
\quo{Load Weapon} sound.  Best sample I ever heard!

Thomas B\"atzler for the encouragement, support, numerous suggestions and
gratuitous two-page review in \quo{Amiga Special}!

Nigel Mansell for winning the British Grand Prix (never mind the World
Championship), giving me a fantastic day out even if I did have to stand
from 4am until 4pm to see it.

This game was brought to you with the aid of:

\begin{itemize}

\item~{}\parbox[t]{14cm}{
    \begin{tabular}{lcl}
      Bruce Springsteen &---& Just about everything\\
      REM               &---& Out Of Time, Green\\
      Cyndi Lauper      &---& She's so Unusual, Night to Remember\\
      Gerry Rafferty    &---& Right Down The Line (Best Of)\\
      The Eagles        &---& Best Of\\
    \end{tabular}}

\item Enough Coke to float a cruise liner (that's the drink, by the way)

\item AMOS and Compiler, both versions 1.34, from Europress Software
      DPaint III

\item Boredom staved off with Civilization, which also gave me the
      inspiration for the \quo{Centre view} function and the keyboard
      shortcuts.

\item And of course, inspiration by the fantastic Rebelstar / Laser
      Squad.
\end{itemize}


\section{Various Rubbish}

If you're the sort of person who likes useless facts, you might be
interested in this bit.

\begin{itemize}

\item The source code for this program is 345K long.
\item Approximately 70K of this is sound and graphics data, the rest is
      just text.
\item It needs a 280K text buffer in the AMOS editor to load it all in.
\item The variable buffer is 85K! (The default is 4K)
\item AMOS takes 50 seconds, after pressing the \quo{Run} button, to actually
      run the program. The delay occurs while it tests the code.
\item Saved as ASCII, it needs 120 sides of A4 to print it out.
\item The hardware setup I used consists of:
      \begin{itemize}
               \item      Amiga A500+ with 2MB
               \item      ICD IDE hard drive + interface
               \item      External floppy
      \end{itemize}

\item V1.0 of the program was begun on April 15th, and finished on September
     1st, 1992. V1.2 was begun on September 5th (!), and finished October
     11th, 1992. It was actually, really, finally finished on November 1st
     1992 (don't ask).                                                     

\end{itemize} 


\section{Background Information}

As I mentioned at the start of this documentation, AoW is \quo{in the tradition
of} Rebelstar and Laser Squad. I understand that many people outside of
England have never heard of either of these, so here's a bit of information
on them.

Both Rebelstar and Laser Squad were written by the same people, the man
behind both being Julian Gollop. Rebelstar Raiders was the first game,
released on the Spectrum around 1983-84, before being re-written and
re-released as Rebelstar later on. This really was a fantastic game, way
ahead of its time, with brilliant graphics and sound (well, for a machine
with a built-in speaker {\bf underneath}, the sound was good)! It was the first
game I know of that used the \quo{action points} system.

Laser Squad followed a few years later, finally making it to the Amiga
around 1989. This was still a brilliant game, and introduced the
line-of-sight feature, the explosive weaponry and the concept of missions.
However, for some reason it just didn't have the charm of its predecessor,
or so I felt anyway.

I understand that Julian is working on Laser Squad 2 at the minute. I hope
this game not only fills the gap between LS 1 and 2, but stands on its own.
I've tried to make it as original as possible, while still obviously being
\quo{strongly influenced} by the other games.

Anyway, I think AoW has a lot to offer of its own, apart from the
influences of these games. Most obviously is the mission editor... it's
possible to do a lot more with this than you would think. Units don't have
to be people, they can be Battlecars, fighter planes, nuclear missiles...
The three missions here are all sci-fi scenarios, but there's no reason why
they have to be. I know of someone who's doing fantasy scenarios. Special
Forces (SAS, Green Berets) are a possibility, and I've heard an unconfirmed
rumour that someone is working on a Disney mission! I always wanted to see
what the Chipmonks could do with a rocket launcher.

I have to mention Space Crusade by Gremlin here. That game has the honour of
being the reason I wrote this. The game is OK, but I couldn't help comparing
it to Laser Squad. It contains several things I hate:


\begin{itemize}

\item Completely random death (\quo{AutoDefences}... aargh!)

\item You can stand right next to an enemy, shoot him with, like,
        a huge Autocannon, and miss him! How realistic. It wasn't
        just that, it's the way that the shot doesn't hit anything
        else, it just disappears if it misses the assigned target.
        Ironically, you can still miss someone from right next to them
        in this game, but at least it's
         a) unlikely, and
         b) fun when the shot hits his team member next to him!

\item Randomly-appearing aliens. You get the feeling that, as the
        game realises it's losing, it just rolls up another alien
        to throw at you.

\end{itemize}

These features are all the result of the game being a conversion of a
boardgame, it's inherently limited. It's still a decent game, though I think
the random deaths spoil it.


\section{The End}

Wow! You didn't read all that rubbish, did you? You're dedicated! Or very
bored.

I always like doc files that rabbit on, like Jeff Minter's epics. I suppose
it sort of gives you an insight into the person that wrote the thing, rather
than the \quo{Press Z to go left} sort of file.

I'm going to stop here before I put the last of the audience to sleep!

Thanks for taking the trouble to look at the game, I hope you like it. I
wrote it rather than get a job over the summer, so somebody better
appreciate it!

Feel free to spread AoW around as much as you like - it's in the Public
Domain, though I keep the copyright. Upload it to bulletin boards, send it
abroad, copy it for your friends.

Play the game, and get your chequebook out!

\vspace{2 cm}

\begin{flushleft}
Dave Smith\\
2.49 pm\\
September 1st 1992\\
~{}\\
Updated October 12th, 1992.\\
Updated again November 1st 1992, though it should really have been done
sooner.\\
\end{flushleft}

\section{A Note From The Editor}

So here I am, left with an empty quarter page to fill. Darn me, I've tried
all the tricks in the book, but this time there's no way to prevent the
text from spilling over on this side, and yet no way to fill it without
stretching the other pages unnaturally. I've tried it, though --- if
you're a fellow \TeX nician you can tell that from my gratitious use of
{\tt $\backslash$pagebreak.}

Anyways, I'll use these last lines to tell you about a forthcoming project:
An illustrated \LaTeX'ed manual for this truly outstanding game. Graphics
will be included by ways of Amiga\TeX's {\tt $\backslash$eps} macro, and the
distribution will also contain PostScript output for the manual, which can
then be printed using Adrian Aylward's PostScript Interpreter. If you don't
already use Post, go and get it --- it's great!

\vfill
\begin{flushleft}
Thomas B\"atzler\\
1.51 am\\
December 13th 1992\\
\end{flushleft}

\end{document}